\documentclass{article}
\def\mytitle{Vijay Saraswat Resume}
%\usepackage{html,htmllist}
\usepackage{epsfig}
%begin{latexonly} 
%\usepackage{url}
\usepackage{fancyheadings}
\setlength{\topmargin}{-0.46in}
\setlength{\oddsidemargin}{-0.0in}
\setlength{\evensidemargin}{-0.0in}
\setlength{\headheight}{.53in}
\setlength{\headsep}{.4in}
\setlength{\footskip}{.4in}
\setlength{\textheight}{8.30in}
\setlength{\textwidth}{6.30in}
\setcounter{tocdepth}{5}                                                        

\pagestyle{fancy}
\usepackage{lastpage209}
\headrulewidth 0.0in
\footrulewidth 0.0in
\rhead{}
\lhead{}
\rfoot{Page \thepage\ of \pageref{LastPage} Saraswat Resume}
\bigskip
\usepackage{ssiarticle}
%\end{latexonly}

\begin{document}
%\begin{htmlonly}
%<p align=right><a href="resume.pdf">printable pdf version</a></p>
%\end{htmlonly}

\begin{center}
{\bf \sc Vijay Anand Saraswat}\\
Mahopac, New York\\
\quad\\
\textbf{Web:} www.saraswat.org\\
\textbf{Email:} vijay@saraswat.org\\
\textbf{IM:} (GMail) vjSaraswat\\
\end{center}

\section*{Education}
\begin{itemize}
\item Ph.D, Computer Science, Carnegie Mellon University, January 1989;
   Title: Concurrent Constraint Programming;
   Advisor: Dana S. Scott.

\item M.S., Computer Science, Carnegie Mellon University, 1985.

\item B.Tech., Electrical Engineering, Indian Institute of Technology,
   Kanpur, India, 1982.
\end{itemize}

\section*{Professional Interests}

I work now at the heart of cognitive computing -- figuring out how to 
build bots that ``know deeply, reason with purpose, learn continuously
and interact naturally.'' Broadly, I am working to broaden logic/programming languages with
machine learning at the core, to better attack messy, real world
(reasoning) problems.  Concretely, I am working on deep natural
language understanding and probabilistic programming: 

\begin{itemize}

\item Natural language understanding: moving from a deep linguistic
  analysis of text to a formal representation that can be used for
  reasoning. Goal is to be able to answer in domains such as legal /
  compliance / financial professional-level questions (e.g. what
  para-legals do today). Current technical approach is to build
  semantic parsers based on constructing semantic forms from the
  output of dependency parsers/LFG parsers using ``glue'', motivated
  by recent work of Reddy et al on ``deplambda''. This speaks to work
  I did at PARC 25 years ago on attacking the syntax/semantics
  interface for natural language, and my work there on knowledge
  representation and reasoning. 

  The enormously exciting ideas on the table now are to augment deep
  linguistically-motivated analyses of natural language utterances
  with machine learning techniques to achieve depth and scale. 

\item  Probabilistic (constraint) programming + machine learning: The
  Achilles' heel of deep learning is the reliance on massive amounts
  of data. My interest is in moving to ``small'' learning
  (particularly for reasoning) -- working on ``theory sketches'',
  approximate logical theories that have ``holes'' in them that can be
  filled by training against data. Again, my focus is on getting to
  robust, scalable, real-world reasoning in complex domains. 
\end{itemize}

To this work I bring a very broad and diverse background in
theoretical computer science (logic, semantics,  algorithms),
programming systems (concurrent, constraint-based computing,
high-performance computing, programming languages) and artificial
intelligence (model-based computing, diagnosis, natural language
processing, machine learning and probabilistic 
logic representations).   

I have worked for over 25 years in Corporate Research (9 years at
Xerox PARC on AI and constraint programming, 4 years at AT\&T Research
on Instant Messaging and Network Communities, and 13 years at IBM TJ Watson on
X10, scale-out programming, analytics and, now, AI). I have also spent 4 years
at three startups (Carnegie Group Inc, Vayusphere, Kirusa). And, I was a
Professor of Computer Science and Engineering at Penn State University
for a year.  

I combine a natural proclivity for vision, intuition, conception and
fundamental research with a passion to develop systems, lead teams and
get the product out the door.  

\section*{Employment and other work}

\begin{itemize}
\item  May 2016 -- onwards -- Chief Scientist, Deep Compliance,
  Cognitive Computing Research at IBM TJ Watson, focused on developing
  a framework for strategic research in the compliance / legal /
  financial spaces. 

\item  January 2015 -- onwards -- Distinguished Research Staff Member,
  reporting to VP, Cognitive Computing Research at IBM TJ Watson.

\item  February 2014 -- January 2015 -- Chief Scientist, Computation
  as a Service Research Division, IBM TJ Watson.

\item 2012 December onwards -- Chief Scientist, IBM Continuous Insight.

\item 2009 -- 2013: Adjunct Professor, Columbia University, NY. Taught
  a course on X10 for five years with Prof Martha Kim.

\item 2009 -- Waseda University, Global COE Visiting Professor.

\item February 2008 -- February 2014: \textit{Manager, Advanced
  Programming Languages}, IBM TJ Watson. 

\item September 2003 -- onwards: \textit{Research Staff Member}, IBM
  TJ Watson. 

\item August 2002 -- Sep 2003: \textit{Professor of Computer Science and
Engineering}, Penn State University. Area: Programming, Languages and Systems.

\item June 2001 -- August 2002: \textit{Vice President, Engineering}, Kirusa.

  Responsible for recruiting the engineering team, developing product
  plans, architecting the product, and delivering the product.
  Product to be trialled with carrier customers in Europe in June
  2002. Work involves development and implementation of XML-based
  Multimodal markup language, for delivery over 2.5G (GPRS)
  networks. Sophisticated integration work, involving WML, VoiceXML,
  HTML, SALT, XML technologies, delivered using a J2EE infrastructure,
  and native PocketPC clients that process speech. Doing significant
  background work on understanding distributed speech recognition, and
  efficient delivery of (possibly processed) speech over GPRS.

  Also responsible for the company's Applications group, Product
  Delivery group, and the company's IT infrastructure. 

  Since August 2002, continuing as a Consultant to Kirusa.

\item December 2000 - June 2001: \textit{Chief Technology Officer}, Vayusphere.

  Responsible for recruiting engg team in San Diego and Mountain View
  (Dev Directors, Architects, Engineers; QA Staff).

  Responsible for defining, architecting and shipping first set of
  products (Monsoon Real-time Messaging Platform), setting the
  technical direction for the company, evaluating vendor technology,
  and parterning with business development.

  Responsible for the ``vision sell'' which brought in the company's
  first major customer.

  Shipped product end of May 2001.
  
\item July 2000 - December 2000:  \textit{Director, Development}, Vayusphere.

  Employee \#6 at Vayusphere. Designed, directed and architected the
  implementation of Vayusphere Instant Messaging. (See ``Systems
  work'' sectionbelow for description of work.)

  Designed and implemented agent architecture used to establish
  client-side extensibility (e.g. integration with Outlook/Exchange)
  and buddy services. The success of this work caused the Board to
  re-center the company around (Wireless) Instant Messaging.
  
  Named on the Times Digital ``Movers and Shakers for 2001'' list (Nov
  2000) for Internet Messaging.
  
\item November 99 - July 2000: 
  \textit{Technology Leader, Instant Messaging and Presence}, AT\&T Shannon
  Labs.

  Secured commitment from AT\&T Wireless to trial AT\&T Instant
  Messaging for their Pocket Net service. 

  Grew the Labs Instant Messaging team. At its peak, the work
  involved over 40 people, with about 15 managers.

  Continued strategic work for AT\&T leadership on Instant
  Messaging. 

  Continued as Co-chair of IETF IMPP WG. Led industry discussions,
  culminating in the formation of the IMUnified group including MSN,
  Yahoo, AT\&T, and other players. Oversaw initial development of the
  IMUnified protocol.

\item November 98 - November 99: 
  \textit{District Manager, Network Community Platform Group}, AT\&T
  Shannon Labs

  Led the design, architecture and development of AT\&T Instant
  Messaging, based on Matrix. Formed the team from scratch. (See
  description of System work below.)

  Managed the growth of the team from a small research-based group to
  a production group incorporating standard AT\&T practices in
  requirements development, system testing, MR management and
  operations management.

  Leveraged an offshore implementation team in India to manage
  low-cost, round-the-clock development efforts.

  Provided strategic input on Instant Messaging to members of the
  AT\&T Operations Group (President of AT\&T Labs, President of
  AT\&T Corporate Strategy, and Chief Counsel for AT\&T). Led
  discussions with industry on interoperable Instant Messaging.

  Obtained \$10m for Instant Messaging and Presence work for 2000
  from the AT\&T Operations Group (chaired by Mike Armstrong), as
  part of the IP Platform work.
  
  Obtained funding from AT\&T Consumer Servics for a Trial of AT\&T
  Instant Messaging.

  Co-Chaired the IETF Working Group on Instant Messaging and Presence
  Protocols (IMPP). All the major industry players -- Microsoft,
  Yahoo, AOL -- have agreed to support the work of this industry
  group.
   
  Industry spokesperson for national media (WSJ, USA Today, NY Times,
  Washington Post, SJ Mercury News ...) on Instant Messaging.

\item September 96 - November 98: 
  \textit{Principal Member, Technical Staff}, AT\&T Shannon Labs
 
  Developed and implemented the underlying ideas for Matrix, an
  architecture for extensible network spaces based on Java. 

  Java evangelist within AT\&T. 

  Identified a major security bug in the design for Java class-loaders
  in JDK 1.1, and worked with Java designers at JavaSoft (Gilad
  Bracha, Sheng Liang) to fix the bug. See their paper in OOPSLA 99.

  Ran Meadows, a school-centered network community (base on MOO
  technology) involving multiple school districts in New Jersey
  and abroad.

\item November 87 --- September 96: 
  \textit{Member, Research Staff}, Xerox Palo Alto Research Center.

  Developed the notion of concurrent constraint programming
  languages.
  
  Initiated and led the thrusts at PARC in model-based
  computing, hybrid computing and network communities.

  (See description of research work below.)

\item April 86 --- November 87: \textit{Artifical Intelligence
  Scientist}, Advanced Product Developments Group, Carnegie Group Inc.

  Helped initiate the ``Carnegie Inference Language'' project,
  to develop the next generation of expert-system development
  tools.  Co-led language design, which was based on ideas from
  production systems, concurrent logic programming and
  object-oriented programming.

\item January 85 --- March 86: \textit{Consultant}, Carnegie Group Inc, Pittsburgh.
 Designed and implemented CRL-Prolog.
\end{itemize}

\section*{Academic awards and Industry recognition}

\begin{description}
\item{2015} ACM SIGPLAN OOPSLA Ten-year ``Most Influential Paper'' award (for
  2005 OOPSLA X10 paper).
\item{2013} IBM Outstanding Scientific Accomplishment for X10
\item{2013} IBM Outstanding Technical Achievement, for contribution to PERCS
\item{2012} HPC Challenge Award for ``Best Performance'' for X10, at SuperComputing 2012.
\item{2009} HPC Challenge Award for ``Best Performance'' for X10 and UPC, at SuperComputing 2009.
\item{2008} HPC Challenge Award for ``Most productive research 
implementation'' for X10 and UPC, at SuperComputing
2008.
\item{2007} HPC Challenge Award for ``Most productive research
implementation'' for X10 (on behalf of X10 team), at SuperComputing
2007.
\item{2004:} ``Most influential paper in 20 years in Concurrent
Constraint Programming'' award from the Association of Logic
Programming for POPL90 paper (w/ Martin Rinard).
\item{2002:} Invited Expert of the W3C Working Group on Multi-Modal
Interaction.
\item{2001:} Co-Chair of the WG on Presence and Instant Messaging
(PRIM), IETF.
\item{1998-2000:} Co-Chair of the WG on Instant Messaging and
 Presence Protocols (IMPP), IETF.
\item{1994:} Excellence in Research Award, Xerox PARC.   
\item{1992:} Excellence in Support of Research Award, Xerox PARC.   
\item{1989:} ACM Doctoral Dissertation Award for the best Computer Science
   PhD Thesis in 1989. Thesis published by MIT Press.
\item{1982:} Ratan Swaroop Gold Medal for all-round excellence,
      Indian Institute of Technology, Kanpur. 
\item{1982:} Scholarship from the Inlaks Foundation for graduate studies
      at the Programming Research Group at Oxford; Fellowship
      offers from Univ. of  Minnesota; scholarship offer from 
      Carnegie-Mellon, Wisconsin-Madison, Brown, etc.  in
      Computer Science.  
\item{1982:} Second prize in All-India Student Paper Contest of the
      Computer Society of India for  Design and implementation of the
      C-code abstract machine for CCN-PASCAL. 
\item{1980:} Merit Scholarship for being the top-ranked student in the
      ``core'' (first 2.5/5) years  at IIT. 
\item{1977:}  Award for ranking in the first 35 (out of over 50,000) in 
      the nation in the All-India Higher Secondary (school-leaving)
      examination. Awarded a National Merit Scholarship by the
      Government of India. 
\item{1977:} Ranked 6th (out of $> 60,000$) in the Joint Entrance Exam
      for all the IITs.  
\item{1977:} Awarded one of five scholarships from India for
      a Baccalaureate at the United World College, Singapore.  
\end{description}

\section*{Research Grants}
\begin{itemize}
\item Principal Investigator on ``Resilient X10'', 2 yr, \$500K grant
  from AFRL, 2013-2014. (IBM Watson Research Lab)
\item   Co-Principal Investigator (with Dave Grove), sub-contracting
  to MIT on DoE X-Stack award ``CAP\^3:  A Computer-Aided Performance
  Programming Platform'', 3 yr, \$1500K, DoE, 2012-2014. (IBM TJ Watson
  Research). DoE extended the award by one year, 2015. 

\item   Co-Principal Investigator (with Danny Bobrow), ``Testing of hybrid
   and reactive systems'', 1 yr, \$150K NASA, 1994-1995. (Xerox PARC)

\item   Principal Investigator, ``Timed Concurrent Constraint
   Programming'', 2 yrs, \$120K, ONR, 1994--1996.  (Xerox PARC)

\item   Co-Principal Investigator (with Danny Bobrow), ``Articulate Spaces:
   Model-based authentic environments for Collaborative Learning'', 2
   yrs, \$2000K ARPA, Department of Defense, 1995-1997.

\item   Co-Principal Investigator (with Danny Bobrow, Billie Hughes, Jim
   Walters), ``Collaborative learning spaces'', NSF, 1 year, 1996-97.
\end{itemize}

In addition, in 1991 Seif Haridi (Director, Swedish Institute of
Computer Science, Stockholm) and I conceived of the \textsc{ ACCLAIM} project
(Advanced Concurrent Constraint Languages --- Applications,
Implementation and Methodology). The project was funded for several
million ECUs by ESPRIT, the European Community Research Funding
Agency, and involved the Max-Planck Institut and DEC Paris Research
Lab, INRIA, DFKI, SICS, RISC Linz, Universidad Politecnica de Madrid,
Universita di Pisa, Marseille Luminy and Katholieke Universiteit
Leuven. The list of deliverables from the project at
\begin{verbatim}
  http://www.sics.se/ps/acclaim/deliverables/perpartner.html
\end{verbatim}
enumerates approximately a hundred papers on concurrent constraint
programming.

\section*{Teaching}
\begin{itemize}
 \item Designed and taught a senior-level course on ``Principles and
   Practice of Parallel Programming'' at Columbia U. (w/ Prof Martha
   Kim), Fall 2009, 2010, 2011, 2012, 2013. Course is now part of regular CS
   curriculum. Course is organized around X10.
 \item Graduate course on Big Data Applications in X10, U.~Padova, June 2014.
 \item Graduate course on Big Data Analysis in X10, U.~Padova, April 2013.
 \item Lectures on X10, Waseda~U., 2008.
 \item Summer course on X10, U.~Pisa, July 2007.
  \item 2-week summer course on ``Hybrid Constraint Programming'', (w/
  Prof R Jagadeesan) at ESSLLI, Nancy, France, August 2004.
  \item CS 598f ``Concurrent Constraint Programming as a Foundation
  for Model-based Programming'', Graduate course, Penn State, Spring
  2003.
  \item CS 428 ``Programming Languages'', Penn State, Fall 2002.
\end{itemize}
\section*{Theses}
\subsection*{Thesis committee member}
\begin{itemize}
\item Stefan Muller (PhD exp. 2017, CMU Computer Science)(Advisor: Umut Acar)
\item Arvind Neelkanthan (PhD exp. 2017, U Mass, Amherts)(Advisor:  Andrew McCullum)
\item Laura Tittolo (PhD 2014, U. Udine) (Advisor: Marco Comini)
\item Sophia Knight (PhD 2013, LiX Polytechnique) (Advisor: Frank Valencia, Catuscia Palamidessi)
\item Carlos Olarte (PhD 2009, LiX Polytechnique) (Advisor: Frank Valencia)
\item Venkatesh Mysore (PhD 2005, New York University) (Advisor: Bud Mishra)
\item Paul Ruet (PhD 1996, University of Paris) (Advisor: Francois Fages)
\item Eric Torng (PhD 1994, Stanford University) (Advisor: Rajeev Motwani)
\item F~Rossi (PhD 1993, University of Pisa) (Advisor: Ugo Montanari)
\end{itemize}

\subsection*{Thesis adviser}
\begin{itemize}
\item  Clifford Tse, ``Linear Janus'' (Master's thesis 1992, MIT Laboratory for
  Computer Science) MIT-PARC VI-A student
\end{itemize}

\section*{Books}
\begin{itemize}
\item Proceedings of ASIAN 03, ed.{} Vijay Saraswat, Springer Verlag
  LNCS 2896, 2003.
\item  Constraint Programming: The Newport Papers, ed.{} Vijay
  Saraswat and Pascal van Hentenryck, MIT Press, 1995.
\item   Proceedings of the 1991 International Symposium on Logic
  Programming, ed.{} Vijay Saraswat and Kazunori Ueda, MIT Press,
  1991.
\item   Vijay Saraswat ``Concurrent Constraint Programming'' MIT
  Press Logic Programming and Doctoral Dissertation Award Series,
  1991.
\end{itemize}
\section*{Edited journal issues}
\begin{itemize}
\item  Vijay Saraswat, Pascal van Hentenryck Special issues of
  CONSTRAINTS on Strategic Directions in Constraint Programming,
  February 1997.
\item  Pascal van Hentenryck, Vijay Saraswat, ed. ``Constraint
  Programming'', in Special Issue of ACM Computing Surveys on
  Strategic Directions in Computer Science, February 1997.
\end{itemize}

\section*{Systems work}
\begin{description}
\item[2012-onwards] Initiated the Resilient X10 project -- extending
  X10 so it is able to internalize node failure, and continue to
  operate. 

\item[2011-onwards] Let the META research project on middleware for
  events, transactions and analytics. This has led to work on an IBM
  product, Continuous Insight. Analytics Architect for CI, and also
  Chief Scientist for CI, contributing to overall architecture and
  Rule language design.

\item[2010-onwards] Initiated the M3R project -- a re-implementation
  of Hadoop in X10, keeping  data in main memory across a
  cluster. Integrated into IBM SPSS products. 

\item[2004-onwards] Leader of the \textsf{ X10} programming language
being designed and implemented at the IBM TJ Watson Research
Center. Author of the \textsf{X10} language manual, principal
implementer of the initial version of the compiler.

 \item[2002-onwards] \textbf{The Java concurrent constraint programming
system} Implementing the \textsf{ jcc} system under the Lesser GPL
licence on SourceForge. The system implements the (default) (timed)
concurrent constraint programming framework in Java, for embedded
and discrete- and continuous-time computing. The implementation is
targeted for model-based programming applications in the NASA domain
(Mars rovers), and in the systems biology domain.

 \item[2001-2002] \textbf{Kirusa Multimodal Platform.}

 As VP Engineering, leading the team that is developing and delivering
 KMMP.  Chief architect for the product, which supports the development
 of multimodal applications (those which simultaneously use voice and
 visual interfaces) for wireless devices. The product is primarily
 based on Java proxy/servlet technology and is designed for 2.5G
 networks (e.g.{} GPRS), and processes the XML-based multimodal markup
 language (M3L), which integrates the WAP Markup Language (WML) and the
 voice markup language (VoiceXML).

 Chief designer of M3L, and author of the language spec.

\item[2000-2001] \textbf{Vayusphere Monsoon Real-time Messaging System}.

Director of the group, designer and chief server-side implementer.
  Monsoon features a scalable, open, extensible architecture for
  Instant Messaging, with two-way interop with email. All server-side
  code was in Java 2; the system also used an Oracle Database, and
  made extensive use of open source components (Apache, Tomcat
  (servlet runner), James (mail engine)). Designed a simple wire
  protocol which is used for all server-to-server and client-to-server
  communication.

  Supervised development of native clients in C and C++ for the RIM
  Mobitex devices, wireless Palm OS devices, and wireless BREW
  devices.  Supervised development in Delphi of a full-featured
  Windows desktop client. Supervised development of HDML/WAP clients
  for phones.

  Assembled the engineering and QA team from scratch, and had the
  product ready for beta in under three months.  Development and QA
  teams were scattered over Mountain Lakes (NJ), Mountain View (Ca),
  San Diego (Ca), Oakland (Ca) and Indiana, and at peak numbered
  approx 20 people.

  Product released to first customer (major national ISP) in nine
  months.

\item[1999-2000] \textbf{AT\&T Instant Intercom.} 

  Founded, staffed and directed the development team that delivered
  AT\&T's instant messaging system into trial with AT\&T Wireless in
  March 2000. I was the main designer and architect, and chief
  server-side implementer. After I left AT\&T, the system was deployed
  by AT\&T Worldnet. In production use today, it has been downloaded
  by over 120,000 users, and supports approx 20,000 simultaneous
  users. It has been designed to scale to a million registered and
  100,000 simultaneous users.

  The system features a modular architecture with several different
  kinds of servers (IM servers, Presence servers, Chat room
  servers). I designed the binary extensible protocol (documented in
  an IETF Draft) used by servers to communicate with each other and
  with clients. Supervised the design of a C++ I/O module to
  circumvent scalability limitations of Java's synchronous I/O
  architecture.

  Supervised development of a native Windows client in Delphi, a Java
  applet client, a black-phone voice client (using PML, a precursor to
  VoiceXML), and of an HDML client (in collaboration with
  AT\&T Wireless).

\item[1997-1998]   \textbf{AT\&T Matrix.}

  Built the first version of the Matrix system in
  Java 1.1.  AT\&T Matrix was a persistent network community server that
  could be extended by programming world objects in Java.  A Matrix
  server is accessed through a Matrix client (developed in Java using
  Swing).  Servers and clients communicate through a completely
  self-contained \textit{session-oriented Remote Method Invocation} (SMI)
  system designed and implemented after analyzing the very poor design
  of the initial Java RMI system.  SMI uses Java's native object
  serialization and class loaders.  World objects provided their own
  graphical user interface, displayed in clients.

  Supervised development of a Java Swing-based client, and the
  development of several world objects (games, bulletin boards, chat
  rooms).  Core team grew to approx. half a dozen people, before
  project was converted into an Instant Messaging project (see below).

  The system was demonstrated to several people in the Labs, was in
  use by the development team, and served as the basis for an internal
  Instant Messaging trial for several hundred people in AT\&T Labs in
  summer 1999.

  The success of this work led to the invitation to form the Network
  Communities Platform Group (NCPG) in AT\&T Labs in November 1998.

\item[1997]\textbf{Meadows.}

 Founded the Meadows online community for parents, school-children and
 teachers across multiple school districts. Responsible for running
 the MOO, and extending it with various objects.

 Taught classes in elementary and middle school (2d grade to 6th
 grade) on Meadows.


\item[1995]\textbf{Pueblo.}

Founded the Pueblo online community (pueblo.xerox.com 7777), using the
 MOO system, with Jim Walters and Billie Hughes of Phoenix College,
 January 1995. As of May 1996, the community had over 1000 characters,
 including over 300 students from Phoenix and New York, and has over
 20,000 objects.

 Responsible for running the MOO, and creating many world objects.

\item[1992-1994]\textbf{Model-based Computing.} 

While at Xerox PARC, initiated and led a project on constraint-based
  machine control.  Implemented the first version of a
  constraint-based real-time scheduler in C++, and supervised a small
  team of engineers at Xerox Engineering in Rochester (New York).
  Code was shipped in the embedded controller for Xerox' digital
  mid-volume reprographics engines.  This work led to several US
  Patents, and an award for Excellence in Research from Xerox PARC.

\item[1992]\textbf{QD-Janus}

Collaborated with Saumya Debray on the QD Janus implementation. See
  ``S.K.~Debray, QD-Janus: A sequential implementation of Janus in
  Prolog, Software Practice and Experience, Volume 23, Number 12,
  December 1993, pp. 1337-1360.''


\item[1986]\textbf{CRL-Prolog, CRL-OPS}

While a scientist at Carnegie Group Inc., designed and implemented
  CRL-Prolog --- a fast Prolog implementation that compiles into
  Common Lisp and is closely integrated with the frame-based language
  CRL.  Supervised the implementation of CRL-Ops.  CRL-Prolog
  and CRL-Ops were shipped as part of the product \textit{Knowledge
  Craft}.


\end{description}

\section*{United States patents}
{\em Please see online lists with the patent office for uptodate information.}
\subsection*{IBM}
\begin{itemize}

% one with Kawachiya
\item US Patent 20150254558 ``Global Production Rules for Distributed
  Data'', with Matthew Arnold, Martin Hirzel, Avraham Shinnar, Jerome Simeon,
Lionel Villard
\item US Patent US 914 7373 B2 ``Transparent efficiency for in-memory
  execution of map reduce job sequences'', with David Cunningham, Ben
  Herta, Avi Shinnar.

\item US Patent 8,924,946 on ``Systems and methods for automatically optimizing high performance computing programming languages'', with Ganesh Bikshandi, Krishna Nandivada Venkata and Igor Peshansky, dated December 30, 2014

\item US Patent 8,726,238 on ``Interactive, iterative program parallelization based on dynamic feedback'', with Robert Fuhrer and Evelyn Duesterwald, dated May 13, 2014.

\item US Patent 8,869,155 on ``Increasing parallel program performance for irregular memory access problems with virtual data partitioning and hierarchical collectives'', with George Almasi, Guojing Cong, David Klapecki, dated Oct 21, 2012.

\item US Patent 8,266,394 on ``Methods for single-owner multi-consumer work queues for repeatable tasks''. with Maged Michael and Martin Vechev, dated Sep 11, 2012.

\item US 20120304178 A1 ``Concurrent reduction optimizations for thieving schedulers''
\end{itemize}

\subsection*{Kirusa}
Several patent applications have been filed by Kirusa based on my work 
in multi-modal systems.

\begin{itemize}
\item US Patent 7,275,217, 2007 on ``System and method for multi-modal browsing with integrated update feature'', with Vijaybalaji Prasanna, Rohitashva Mathur and Shirish Vaidya, on Sep 25, 2007.
\end{itemize}

\subsection*{AT\&T}
\begin{itemize}
\item US Patent 20020071539 A1 on ``Method and apparatus for telephony-enabled instant messaging''.
%\item Patent EP1119168 on ``Internet call waiting service for wireless connections'' 

\item Patent pending on ``Multi-modal directories for telephonic applications'', WO 2003015388 A1.
\end{itemize}

\subsection*{PARC}
Patents resulting from model-based scheduling work at Xerox PARC:
\begin{itemize}
\item   US 5,831,853  ``Automatic construction of digital
  controllers/device drivers for electro-mechanical systems using
  component models'', 11/98

\item   US 5,701,557 ``Machine graphs and capabilities to represent
  document output terminals composed of arbitrary configurations'', 12/97

\item  US 5,696,893 ``System for generically describing and
  scheduling operations of modular printing machine'', 12/97

\item  US 5,631,740 ``Transducers with constraints model for print
  scheduling'', 5/97

\item  US 5,617,214 ``Commitment groups to generalize the scheduling
  of interdependent document output terminal capabilities'', 4/97

\item  US 5,504,568 ``Print sequence scheduling system for duplex
  printing apparatus'', 4/96
\end{itemize}


\section*{Publications}
{\em Please see Google  Scholar for up-to-date  lists.}

\section*{AI, Logic, Knowledge Representation and Reasoning}
\begin{itemize}
\item C~Cornelio, V~Saraswat ``Expressing Probabilistic Grahical
  Models in RCC'', AAAI 2017 Workshop on Symbolic Inference and
  Optimization. 

\item  A~Loreggia, H~Samulowitz, Y~Malitsky, V~Saraswat ``Deep
Learning for Algorithm Portfolios'', AAAI 2016.  

\item V~Saraswat and J~Milthorpe ``The Continuous Allreduce algorithm
  for asynchronous stochastic gradient descent'', NIPS 2015  Workshop
  on Non-Convex Optimization for Machine Learning: Theory and Practice.

\item C~Cornelio, A~Loreggia, V~Saraswat ``Logical
Conditional Preference Theories'', MPREF workshop, IJCAI 2015. 

\item U~Grandi, A~Loreggia, F~Rossi and V~Saraswat. A
Borda Count for Collective Sentiment Analysis. Annals of Mathematics
and Artificial Intelligence, special issue on ``Preferences and
Computational Social Choice'', 2015. 

\item U~Grandi, A~Loreggia, F~Rossi and V~Saraswat. ``From Sentiment
  Analysis to Preference Aggregation''. In Proceedings of the 2014
  International Symposium on Artificial Intelligence and Mathematics
  (ISAIM-2014), 2014.   

\item R~Jagadeesan and G~Nadathur and V~Saraswat
``Testing concurrent systems: An interpretation of intuitionistic
logic'', Proceedings of FST\&TCS, December 2005.

\item  M~Fromherz, V~Saraswat and D~Bobrow
  ``Model-based computing: Developing flexible machine control
  software'', Artificial Intelligence, 114(1-2): 157-202 (1999)

\item  V~Gupta, R~Jagadeesan and V~Saraswat ``Probabilistic
  Concurrent Constraint Programming'', Proceedings of the
  International Conference on Concurrency Theory, LNCS, CONCUR '97,
  243-257.

\item  M~Fromherz, V~Gupta and V~Saraswat, ``CC
     --- A  generic framework for domain specific languages'',
    Workshop  on Domain-oriented specification languages, POPL 97.

\item  M~Fromherz and V~Saraswat   ``Model-based
  computing: using concurrent constraint programming for modeling
  and model compilation'', U. Montanari  and F. Rossi (ed.)
  Principles and Practices of Constraint Programming, CP'95,
  Springer-Verlag, LNCS 976, Sep 1995, pp. 629--635. 

\item  Y~Iwasaki, A~Farquhar, V~Saraswat, D~Bobrow
  and V Gupta ``Modeling time in hybrid systems: How fast is
  ``instantaneous''?'', Proceedings of the International Joint
  Conference on Artificial Intelligence, Montreal, August, 1995.

\item H~Wong and M~Fromherz and V~Gupta and
  V~Saraswat. ``Control-based programming of electro-mechanical 
  controllers.'' Proceedings of the IJCAI Workshop on Executable
  Temporal Logics, Montreal, August 1995.

\item  V~Gupta and V~Saraswat and P~Struss, ``A model
  of a photocopier paper path'', Proceedings of the 2nd IJCAI
  Workshop on Engineering Problems for Qualitative Reasoning,
  August 1995.
  
\item  M~Fromherz and V~Saraswat ``Model-based
  computing: constructing constraint-based software for
  electro-mechanical systems'' Practical Applications of
  Constraint Technology, Paris, France, April 1995, pp. 63-66.

\item M~Fromherz, D~Bell, D~Bobrow, B~Falkenhainer, V~Saraswat and
  M~Shirley ``\textsc{ Rapper}: The Copier Modeling Project'', Working
  Papers of the Eight International Workshop on Qualitative Reasoning
  about physical systems'', pages 1-12, June 1994. 

\item O~Raiman and J~de Kleer and V~Saraswat ``Critical
  reasoning'', Proceedings of the International Joint Conference on
  Artificial Intelligence, 1993.

\item P~Codognet and V~Saraswat, ``Abduction in Concurrent
  Constraint Languages'', Proceedings of the First Compulog Network
  meeting on Logic Programming and Artificial Intelligence, London,
  U.K., 1992.

\item  O~Raiman and J~de Kleer and V~Saraswat and
  Mark Shirley ``Characterizing non-intermittent faults'',
  Proceedings of the National Conference on Artificial
  Intelligence, June 1991.
  
\item V~Saraswat, J~de Kleer and O~Raimon
  ``Contributions to the theory of diagnosis'', International
  Workshop on Principles of Diagnosis, Stanford University, July
  1990.

\item V~Saraswat ``CP as a general-purpose constraint-language'', AAAI 1987.
\end{itemize}

\subsection*{Natural language}
\begin{itemize}

\item Introduction. (M~Dalrymple, F~Pereira, John Lamping,
   V~Saraswat) In \textit{Semantics And Syntax in Lexical Functional
   Grammar: The Resource Logic Approach} edited by M~Dalrymple. The
   MIT Press, 1999.

\item LFG qua Concurrent Constraint Programming. (V~Saraswat) In
   \textit{ Semantics And Syntax in Lexical Functional Grammar: The Resource
    Logic Approach} edited by M~Dalrymple. The MIT Press, 1999.

\item Relating Resource-based semantics to categorial semantics. (Mary
    Dalrymple, V~Gupta, John Lamping, and V~Saraswat) In
   \textit{ Semantics And Syntax in Lexical Functional Grammar: The Resource
    Logic Approach} edited by M~Dalrymple. The MIT Press, 1999.

\item    Quantification, Anaphora and Intensionality (Mary
    Dalrymple, John Lamping, F~Pereira, and V
    Saraswat) In \textit{ Journal of Logic, Language and Information 6
      (3), pp 219-273, July 1997.}

\item    A Deductive Account of Quantification in LFG.  (Mary
    Dalrymple, John Lamping, F~Pereira, and V
    Saraswat) In \textit{ Quantifiers, Deduction, and Context}, ed.\
    Makoto Kanazawa, Christopher J.~Pi\~{n}\'{o}n, and Henriette
    de~Swart.  Stanford, California: Center for the Study of
    Language and Information.  1996. 

\item    Andrew Kehler, M~Dalrymple, John Lamping, and V~Saraswat
    The Semantics of Resource Sharing in Lexical-Functional Grammar.  
    Proceedings of the 1995 Meeting of the
    European Chapter of the Association for Computational
    Linguistics, Dublin, Ireland. March 1995.

\item    M~Dalrymple, John Lamping, F~Pereira, and V~Saraswat
    Intensional Verbs Without Type-Raising or Lexical Ambiguity.
    In Logic, Language and Computation, volume 1, ed. Jerry
    Seligman and Dag Westerstaahl. Stanford, California:
    Center for the Study of Language and Information.  1996.
    Also in Proceedings of the Conference on
    Information-Oriented Approaches to Logic, Language and
    Computation/Fourth Conference on Situation Theory and its 
    Applications, Saint Mary's College of California, Moraga,
    California. June 1994.

\item  M~Dalrymple, J~Lamping, and V~Saraswat. 1993.
    LFG semantics via constraints. In Proceedings of the Sixth
    Meeting of the European ACL, University of Utrecht, April.
    European Chapter of the Association for Computational
    Linguistics. 
    
\item    M~Dalrymple, A~Hinrichs, J~Lamping, and V~Saraswat.
    The resource logic of complex predicate interpretation.
    In Proceedings of the 1993 Republic of China Computational
    Linguistics Conference (ROCLING), Hsitou National Park,
    Taiwan, September. Computational Linguistics Society of R.O.C.
\end{itemize}

\subsection*{Parallel Constraint Solvers}
\begin{itemize}
\item D Munera, D Diaz, S Abreu, F Rossi, V~Saraswat, P Codognet
  ``Solving Hard Stable Matching Problems via Local Search and
  Cooperative Parallelization'', 29th AAAI Conference on Artificial
  Intelligence, 2015. 

\item D Bergman, A Cire, A Sabharwal, H Samulowitz, V~Saraswat, W Jan van Hoeve: ``Parallel Combinatorial Optimization with
  Decision Diagrams'', CPAIOR, 2014.

\item B Bloom, D Grove, B Herta, A Sabharwal, H Samulowitz, V~Saraswat ``SatX10: A Scalable Plug\&Play Parallel
  SAT Framework'',  in Proceedings of the 15th International Conference on Theory and Applications of
Satisfiability Testing (SAT 2012).   

\end{itemize}

\subsection*{Programming Languages and Systems}
\subsection*{X10}
\begin{itemize}
\item O Tardieu, B Herta, D Cunningham, D Grove, P Kambadur,
  V~Saraswat, A Shinar, M Takeuchi, M Vaziri, W Zhang ``X10 and APGAS
  at Petascale'', ACM Transactions on Paralllel Computing, March 2016 

\item S~Hamouda, J~Milthorpe, P~Strazdins, V~Saraswat ``A
  Resilient Framework for Iterative Linear Algebra Applications in
  X10'', 16th IEEE International Workshop on Parallel and Distributed
  Scientific and Engineering Computing, PDSEC 2015.

\item S~Crafa, D~Cunningham, V~Saraswat, Avraham Shinnar,
  O~Tardieu ``Semantics of (Resilient) X10'', ECOOP 2014.

\item D~Cunningham, D~Grove, B~Herta, Arun Iyengar,
  Kiyokuni Kawachiya, Hiroki Murata, V~Saraswat, Mikio Takeuchi
  O~Tardieu ``Resilient X10: efficient failure-aware
  programming'', PPoPP 2014. 

\item O~Tardieu, B~Herta, D~Cunningham, D~Grove,
  Prabhanjan Kambadur, V~Saraswat, Avraham Shinnar, Mikio
  Takeuchi, Mandana Vaziri ``APGAS at Peta-scale'', PPoPP 2014.

\item Wei~Zhang, O~Tardieu, D~Grove, B~Herta, T~Kamada, V~Saraswat, M~Takeuchi  ``GLB: Life-line based
  Global Load Balancing library in X10'', Workhop on Parallel
  Programming for Analytic Applications, PPoPP 2014.

\item T~Yuki, P~Feautrier, S~Rajopadhye, V~Saraswat
  ``Array dataflow analysis for polyhedral X10 programs'', PPoPP 2013.

\item M~Takeuchi, D~Cunningham, D~Grove, V~Saraswat ``Java
interoperability in Managed X10'', Proceedings of Third ACM SIGPLAN
X10 Workshop, pp 39--46. 

\item O~Tardieu, N~Nystrom, I~Peshansky and V~Saraswat
  ``Constrained Kinds'', OOPSLA 2012.
\item Y~Zibin, D~Cunningham, I~Peshansky, V~Saraswat
  ``Object initialization in X10'', ECOOP 2012.

\item A~Shinnar, D~Cunningham, V~Saraswat, B~Herta
  ``M3R: increased performance for in-memory Hadoop jobs'', VLDB
  2012. 

\item V~Saraswat, P~Kambadur, S~Kodali, D~Grove, S~Krishnamoorthy ``Lifeline-based global load
  balancing'', PPoPP 2011. 

\item D~Cunningham, R~Bordawekar, V~Saraswat ``GPU
  programming in a High-level language: compiling X10 to CUDA'',
  Proceedings of the ACM SIGPLAN X10 workshop, 2011.

\item D~Grove, O~Tardieu, D~Cunningham, B~Herta, I~Peshansky, V~Saraswat ``A Performance Model for X10
  Applications'',   Proceedings of the ACM SIGPLAN X10 workshop, 2011.

\item V~Saraswat, George Almasi, Ganesh Bikshandi, Calin Cascaval,
  D~Cunningham, D~Grove, Sreedhar Kodali, Igor Peshansky,
  O~Tardieu ``The Asynchronous Partitioned Global Address Space
  Model'', AMP'10: Proceedings of the First Workshop on Advanced in
  Message Passing, 2010.
\item Ganesh Bikshandi, Jose Castanos, Sreedhar Kodali, Krishna Nandivada, Igor Peshansky, V~Saraswat, Sayantan Sur, Pradeep Varma, Tong Wen ``Efficient, Portable Implementation of Asynchronous Multi-place Programs'', PPoPP 2009.

\item Maged Michael, Martin Vechev and V~Saraswat ``Idempotent Work stealing'', PPoPP 2009.

\item Nathaniel Nystrom, V~Saraswat, Jens Palsberg and Christian
Grothoff ``Constrained types for OO Languages'', OOPSLA
2008.

\item Guojing Cong, Sreedhar Kodali, Sriram Krishnamoorthy, Doug Lee,
V~Saraswat and Tong Wen ``Solving irregular graph problems using
adaptive work-stealing'', ICPP 2008.

\item Satish Chandra, V~Saraswat, Vivek Sarkar and Ratislav Bodik,
``Type Inference for Locality Analysis of Distributed Data
Structures'', Proceedings of the ACM Symposium on Principles and
Practice of Parallel Programming, 2008.

\item V~Saraswat, R~Jagadeesan, Maged Michael and Christoph
von Praun, ``A Theory of Memory Models'', Proceedings of the ACM
Symposium on Principles and Practice of Parallel Programming, March
2007.

\item Philippe Charles, Christian Grothoff, Kemal Ebcioglu, Allan
  Kielstra, Christoph von Praun, V~Saraswat and Vivek Sarkar
  ``X10: An Object-oriented Approach to Non-Uniform Cluster
  Computing'', Onwards! Track of the Proceedings of OOPSLA 2005.

\item V~Saraswat and R~Jagadeesan ``Concurrent Clustered
  Programming'', Proceedings of CONCUR, 2005.

\item ``Report on the Experimental Language X10'', principal author,
July 2005.
\end{itemize}				   


\subsection*{Semantic foundations for concurrent programming}
\begin{itemize} 
\item V~Saraswat, V~Gupta, R~Jagadeesan ``TCC, With
  History'', Horizons of the Mind, 2014, pp 458-475.

\item Catuscia Palamidessi, V~Saraswat, Frank Valencia and Bjorn
Victor ``On the expressiveness of linearity and persistence in the
pi-calculus'', LICS 2006.

\item V~Saraswat ``Constraint-Based Memory Machines: A framework
for Java Memory Models'', ASIAN 2004, pp 494-508.

\item V~Saraswat and R~Jagadeesan ``Static support for
  capabilities in Java'', Second APPSEM workshop, U of Nottingham,
  April 2003. 


\item  V~Gupta, R~Jagadeesan and V~Saraswat ``Models
  of concurrent constraint programming'', Proceedings of the
  International Conference on Concurrency Theory, LNCS 1119 August
  1996.

\item V~Gupta, R~Jagadeesan and V~Saraswat ``Truly
  Concurrent Constraint Programming'', Theoretical Computer Science,
  Volume 278, pp 223-255, 2002. Conference version in Proceedings of
  the International Conference on Concurrency Theory, LNCS 1119 August
  1996.

\item  Ugo Montanari, F~Rossi, and V~Saraswat ``Event
  structure semantics for concurrent constraint programming'',
  1994.  
  
\item  Ugo Montanari, F~Rossi, and V~Saraswat ``CC
  programs with both in- and non-determinism'', 1994.

\item  Prakash Panangaden, V~Saraswat, Phillip J. Scott and
  Robert Seely, ``A Hyperdoctrinal view of concurrent constraint
  programming'', Proceedings of the REX Workshop on Semantics:
  Foundations and Applications, eds. J. W. deBakker, W.-P. de Rover
  and G. Rozenberg, LNCS 666, pp 457-476, 1993.
  
\item  V~Saraswat ``The category of constraint systems is
  Cartesian-closed'', Proceedings of the Symposium on Logic in
  Computer Science, Santa Cruz, June 1992.

\item R~Jagadeesan, V~Saraswat and Vasant Shanbhogue ``Angelic
  non-determinism in concurrent constraint programming'', Technical
  Report, Xerox PARC, 1992.
  
\item  V~Saraswat and Rinard M. and Panagaden, P. ``Semantic
  foundations for concurrent constraint programming'', \textit{
  Proceedings of the ACM Symposium on Principles of Programming
  Languages}, Orlando, January 1991.
  
\item Patrick Lincoln and V~Saraswat ``Proofs as concurrent
  processes: A logical interpretation for concurrent constraint
  programming'', Technical Report, Systems Sciences Laboratory, Xerox
  PARC, November 1991. (Revised report, Penn State University, April
  2003.)
  
\item  V~Saraswat and Rinard, M. ``Concurrent constraint
  programming'', Proceedings of the ACM Symposium on
  Principles of Programming Languages 1990, San Fransisco, January
  1990.
  
\item  V~Saraswat ``A somewhat logical formulation of CLP
  synchronization primitives'', Proceedings of Logic Programming,
  August 1988, MIT Press.
  
\item  V~Saraswat ``The language \textbf{CP}: Definition and
  Operational semantics'', in Proceedings of the ACM
  SIGACT-SIGPLAN Conference on Principles of Programming
  Languages, Munich, January 1987.
  
\item  V~Saraswat ``\textsf{CP} as a general-purpose
  constraint-language'', Proceedings of the National
  Conference on Artificial Intelligence, (AAAI), Seattle, July
  1987.
  
\item  V~Saraswat ``The language GHC: operational semantics
  and comparison with \textbf{CP(\texttt{ !},\texttt{ |})}'', Proceedings of the
  Fourth IEEE Symposium on Logic Programming, San Fransisco,
  September 1987.
  
\item  V~Saraswat ``Partial Correctness semantics for \textbf{
  CP(\texttt{ !},\texttt{ |}, \texttt{ \&})}'', Proceedings of the Conference on
  Foundations of Software Technology and Theoretical Computer
  Science, Springer Verlag LNCS 206, pp. 347-368, December 1985.
\end{itemize}

\subsection*{Real-time and hybrid systems}
\begin{itemize} 
\item  V~Gupta, R~Jagadeesan and V~Saraswat
  ``Computing with Continuous Change'',  Science of
  Computer Programming, 30 (1:2) 3-49, 1998.

\item  V~Saraswat, R~Jagadeesan and V~Gupta
  ``Timed Default Concurrent Constraint Programming'', In Journal of
  Symbolic Computation 22 (5,6) 475--520, 1996. Extended abstract
  published in the Proceedings of the ACM Symposium on Principles of
  Programming Languages, San Fransisco, 1995.
  
\item  V~Gupta, R~Jagadeesan and V~Saraswat
  ``Hybrid CC, Hybrid Automata, and Prgram Verification'', Hybrid
  Systems Workshop, DIMACS, Rutgers, October 1995. Appeared in Hybrid
  Systems III Verification and Control ed.{} R. Alur, T.A.{}
  Henzinger, E.D.{} Sontag (LNCS 1066), Springer-Verlag, Berlin, 1996.

\item  V~Gupta, R~Jagadeesan, V~Saraswat, and Daniel
  Bobrow ``Programming in Hybrid Constraint Languages'', Hybrid
  Systems Workhsop, Cornell, October 1994. Hybrid Systems II, ed.{}
  P.~Antsaklis, W.~Kohn, A.~Nerode, S.~Sastry (LNCS 999),
  Springer-Verlag, Berling, 1995.

\item  V~Saraswat, R~Jagadeesan and V~Gupta
  ``Foundations of Timed Concurrent Constraint Programming'',
  Proceedings of the Symposium on Logic in Computer Science, Paris,
  July 1994.
  
\item  V~Saraswat, R~Jagadeesan and V~Gupta
  ``Programming in Timed Concurrent Constraint Programming'',
  Chapter in Constraint Programming, ed.{} B. Mayoh and E. Tyugu,
  NATO ASI Workshop, April 1994.
\end{itemize}

\subsection*{Concurrent programming languages and paradigms}
\begin{itemize} 

\item V~Saraswat, R~Jagadeesan and V~Gupta ``jcc:
   Integrating timed default concurrent constraint programming into
   java'', Proceedings of the Eleventh Portugese Conference on
   Artificial Intelligence (EPIA '03), Springer Verlag LNCS, December
   2003.

\item V~Saraswat ``Java is not type-safe'', Web-note
  \texttt{http://www.research.att.com/~vj/bug.html}. Described a major
  bug in the design of the class-loader mechanism. This was
  acknowledged and fixed by Sun in a major redesign. See Bracha and
  Liang's paper in OOPSLA 98.

\item Pascal van Hentenryck, Yves Deville, V~Saraswat ``Design,
    implementation and evaluation of the constraint language cc(FD)'',
    Journal Of Logic Programming 37(1-3):139-164 (1998). Conference
    paper in LNCS 910, pp 293-316 (1994).

\item V~Saraswat and Patrick Lincoln ``Higher-order, linear
    concurrent constraint programming'', Xerox PARC Technical report,
    August 1992.

\item V~Saraswat and Kenneth Kahn and Jacob Levy ``\textsf{
    Janus}: A step towards distributed constraint programming'', \textit{
    Proceedings of the North American Conference on Logic
    Programming}, Austin, Texas, October 1990.

\item    Kenneth Kahn and V~Saraswat ``Actors as a special
    case of concurrent constraint (logic) programming'', \textit{
    Proceedings of the ECOOP/OOPSLA conference, 1990}.


\end{itemize}

The \textsf{Janus} work has significantly influenced the development of
the (distributed, secure) language \texttt{ E}, currently being developed
by an open source group (\texttt{ www.erights.org}), and partly funded by
a DARPA grant to Mark S. Miller.

\subsection*{Constraint programming in Software Engineering}
\begin{itemize} 

\item R~Jagadeesan, Will Marrero, Corin Pitcher and V~Saraswat
``Timed Constraint Programming: A Declarative Approach to Usage
Control'', Proceedings of Principles and Practices of Declarative
Programming, June 2005.

\item    V~Saraswat ``Compositional Computing'', CONSTRAINTS
    2(1):95-97 (1997)

\item F~Rossi and V~Saraswat ``Constraint Programming'',
in Encyclopedia of Computer Science and Technology (entry: Constraint
Programming), A. Kent and J.G.{} Williams eds, Marcel Drekker Inc,
1994.

\end{itemize} 

\subsection*{Program Sketching}
\begin{itemize} 

\item Armando Solar-Lezama, Gilad Arnold, Liviu Tancau, Ratislav
Bodik, V~Saraswat and Sanjit Seshia ``Sketching Stencils'', in ACM
SIGPLAN Conference on Programming Language Design and Implementation
(PLDI '07).

\item Armando Solar-Lezama, Liviu Tancau, Ratislav Bodik, V
Saraswat ``Combinatorial Sketching for Finite Programs'', in ASPLOS
2006.
\end{itemize} 

\subsection*{Concurrent programming: techniques, algorithms}
\begin{itemize} 

\item   Rajeev Motwani, Suresh Venkatsubramaniam, Rina Panigrahy, V
   Saraswat ``On the decidability of accessibility problems'', ACM
   Symposium on the Theory of Computing, 2000.

\item   Eric Torng, Rajeev Motwani, and V~Saraswat ``Online
   scheduling with lookahead: Multipass assembly lines''
   INFORMS Journal on Computing, 1998.

\item   Saraswat, V.A. et al. ``Detecting stable properties
   of networks in concurrent logic programming languages'', in
   Proceedings of the ACM Conference on Principles of
   Distributed Computing, Toronto, August 1988.
  
\item   V~Saraswat ``Merging many streams efficiently: the
   importance of atomic commitment'', chapter in ``Concurrent
   Prolog: Collected Papers'', ed. E. Shapiro, MIT Press, December
   1987.
\end{itemize}

\section*{Applications}
\subsection*{Visual programming}
\begin{itemize} 
\item    Kenneth M. Kahn and V A. Saraswat ``Complete
    visualization of concurrent programs and their execution'',
    Proceedings of the IEEE Workshop on Visual Programming, October
    1990. 
\end{itemize} 

This work led to a rich body of work on Visual Programming, cf.{}
Pictorial Janus systems, and also to the company \textit{ Animated
Programs} founded by Ken Kahn. The company has introduced a
revolutionary product for school-children ``ToonTalk'', in the
tradition of Logo. See \texttt{ www.toontalk.com}.

\subsection*{Multi-modal Systems}
\begin{itemize}
\item Stephane Maes and V~Saraswat ``Multi-Modal Requirements'', 
W3C Note, January 2003.
\end{itemize}				   

\subsection*{Network communities}
\begin{itemize}
\item V~Saraswat and F~Pereira ``Interaction media: Some
  thoughts on models for cyberspace'', Proceedings of the Virtual
  Worlds in Simulation Conference, San Francisco, January 1999.

\item  V~Saraswat ``Design requirements for network spaces'',
  Proceedings of the Virtual Worlds in Simulation Conference, San
  Francisco, January 98.

\item Jay Carlson, Roger Crew, Ken Fox, Richard Goddard, Dave Kormann,
  Erik Ostrom, John Ramsdell, V~Saraswat, Andrew Wilson ``The MUD
  Client Protocol, Version 2.1'', http://www.moo.mud.org/mcp2.

\item  V~Saraswat ``The dog, the catcher, the fish and the
  frying pan: Melding work, play and theater in network
  community'', Virtual Communities 97, February 1997, Sydney,
  Australia. 
  
\item  Vicki O'Day, Daniel Bobrow, Billie Hughes, Kimberly Bobrow,
  V~Saraswat, JoAnne Talazus, Jim Walters, Cynde Welbes
  ``Community Designers'', Participatory Design Conference, 1996.

\item  Daniel Bobrow, Vicki O'Day, V~Saraswat, Billie Hughes and
  Jim Walters ``Learning through computationally-mediated
  conversations: Interaction and Construction in virtual
  spaces'', Presented at the Annual Meeting of the American
  Anthropological Association, Washington D.C., November 1995.
\end{itemize}

    

\section*{Professional activities}

%\subsection*{Invited Panels:}
%\begin{itemize}

%\item ``Language Innovations for High-Productivity Computing
%Systems'', PPoPP 2005, Chicago.

%\item  ``The Future of Constraint Programming'', CP 96, Boston, 1996.
%\end{itemize}

\subsection*{Selected Invited Presentations}
\begin{itemize}
\item Invited talk at ILP'16 (Inductive Logic Programming), London, Sep 2016. 
\item Invited talk at DISCO'16, Crete, Greece, June 2016.
\item Invited talk on X10 at U Pennsylvania, Feb 2015.
\item Invited talk on ``Writing Robust Applications in Resilient X10'' at ACSI 2015, Tsukuba, Jan 2015.
\item Invited talk at Exascale Runtime workshop on Resilient X10 at TU Munich, December 2014.
\item Invited talk on Resilient X10 at ETH Software Correctness and Reliability Worshop, Zurich, October 2014. 
\item Participant in School on ``Constraints, Data and Optimization'', Dagstuhl, October 2014. 
\item Invited talk at CP'14, Lyon, September 2014.
\item Keynote on ``Computing in the post-cloud era'' InForum, Porto, Portugal, September 2014.
\item Invited talk on ``Programming in X10'', Department of Engineering, Padua University, June 2014.
\item Invited talk on ``C10: Probabilistic Concurrent Constraint Programming'' at Software Day, Tsinghua University, April 2014. 
\item Invited talk on ``Constraints Solvers in X10'' at CPAIOR 2013.
\item Invited talk on X10 at AICS International Symposium, Kobe, February 2013.
\item Invited talk at Parallel Constraint Solver workshop at Shonan, Tokyo, May 2012.
\item Invited talk at Yonezawa Festschrifft, Kobe, May 2012.
\item Invited talk at PMCS Workshop, International Conference on
  Constraint Programming (CP'11), Perugia, Sep 2011.
\item Keynote at Ohio LinuxFest 2011 on ``PRogramming in the
  Concurrency Era: The X10 Programming Language'', Columbus, Sep 2011.  
\item Keynote at LCPC on ``Constrained types: What they are, and what
  they can do for you'', Fort Collins, Sep 2011. 
\item PGAS Tutorial at Supercomputing 2008 (with Tarek el-Ghazawi and Brad Chamberlain).
\item Lectures at Waseda University as Global COE Visiting Professor,
April 2008.
\item ``X10: Programming Parallel machines, productively'', Keynote
talk at ASPLAS 2007, November Singapore.
\item X10 tutorial at PACT 2006, OOPSLA 2006, PPoPP 2007, PLDI 2007.
\item X10 lectures at ``Trends in Concurency'' Summer School at
Bertinoro, Italy.
\item ``X10: An Object-Oriented Approach to Non-uniform Computing'',
IBM PL Day, April 2005.
\item ``Hybrid Synchronous Languages'', NYU, March 2005.
\item ``Hybrid Constraint Programming'', ETAPS Workshop on
``Synchronous Languages, Applications and Programming'', Barcelona,
March 2004.
\item ``The Emergence of Systems Biology'', NJPLS February 2004.
\item 
Penn State Logic Seminar, ``The logic of concurrent constraint
programming'', April 2003.
\item 
Invited participant, CUE Workshop on Component-based systems, Macau,
China SAR, October 2002.
\item IBM Research (T.J.~Watson Research Lab) ``Java as a metalinguistic
framework'', September 2002.
\item 
IBM Research (T.J.~Watson Research Lab) ``The design of M''
October 2002.
\item
Chair, session on Multimodal Instant Messaging (Pulver Presence and
Instant Messaging Industry meeting), October 2001.

\item
Chair, session on Wireless Instant Messaging (Pulver Presence and
Instant Messaging Industry meeting), October 2000.

\item Invited Plenary Talk at ``Instant Messaging 2000'', Boston, May
    2000.

\item Tutorial on Concurrent Constraint Programming at CONCUR
    '96, Pisa Italy.

 \item Invited Talk at Workshop on Language, Logic and
    Information, CSLI, May 1996.

 \item Stanford University Theory Group, ``Hybrid Constraint
    Programming'', May 1996. 

 \item Distinguished Seminar in Programming Languages, U. of
    Chicago,  May 1996 ``Virtual World Programming Languages''. 

 \item Invited talk, First International Conference
    on Constraint Programming, Cassiss, France, September 1995. 

 \item Invited to attend Institute on Semantics of Programming
    Languages at Newton Institute, Cambridge, August 1995.  

 \item Keynote talk, First International Workshop on
    Concurrent Constraint Programming, Venice, May 1995. 

 \item Invited Talk on Timed Concurrent Constraint
    Programming, Royal Institute of Mathematics, Kyoto, August
    1994. 

 \item Workshop on Information Systems, Cosener House, Oxford,
    March 1992 (organized by Samson Abramsky and Bill Rounds).

 \item Sixth Italian Logic Programming Conference (GULP '91)
    on ``The Past as Prolog: Beyond Logic Programming''.

 \item Tutorial on ``The semantics of concurrent constraint
    programming languages'' at the North American Logic Programming
    Conference, Austin, Texas, October 1990.

 \item Tutorial on ``The paradigm of concurrent constraint
    programming'', at the Seventh  International Conference on Logic
    Programming, Jerusalem, June 1990. 

 \item ``Gigalips Workshop'', Swedish Institute of Computer
    Science, Stockholm, April 1989 (organized by Seif Haridi).

 \item Workshop on ``Languages and Constraints''  University of
    Rhode Island, April 1988 (organized by J. Cohen and J-.L.Lassez).

 \item AAAI-sponsored workshop on ``Concurrent Logic Programming
    and Open systems'', Xerox PARC, September 1987 (organized by Ken
    Kahn).

 \item Talks on various aspects of concurrent constraint programming
    at CMU, Cornell, MIT, Yale, UC Berkeley, Oxford University, U.~of
    Edinburgh, Imperial College, University of Texas at Austin, AT\&T
    Bell Labs, U.~of Pennsylvania, IBM T.J.~Watson Research Center,
    U.~of Utah, U.~of Arizona, U.~of Pisa, INRIA Versailles, Swedish
    Institute of Computer Science, Bristol Univ., Penn State
    University etc.
\end{itemize}

\subsection*{Refereeing responsibilities}
\begin{itemize}
\item  Journal of Logic Programming; 
\item  Artificial Intelligence; 
\item  Journal of the Association for Computing Machinery; 
\item  Theoretical Computer Science; 
\item  IEEE Computer;
\item  IEEE Software;
\item National Science Foundation (multiple panels)
\item DoE ASCR (multiple panels)
\item Science of Computer Programming
\item IEEE Transactions on Computers;
\end{itemize}

\subsection*{Editorial and Review Responsibilities}

\begin{itemize}
\item Program Committee, AAAI 2016, PPoPP 2016.
\item Program Committee, PPoPP 2014, ParSearchOpt 2014, EuroPar 2014.
\item Program Committee, PPoPP 2013, SuperComputing 2013.
\item Program Committee, OOPSLA 2010, PPoPP 2010; External Program
  Committee, PLDI 2010.
\item Program Committee, LICS 2009.
\item Program Committee, Coordination, 2008; Places 2008.
\item Program Committee, Compiler Construction, 2007.
\item Program Committee, European Symposium on Programming, 2007.
\item Program Committee, HiPC, 2007.
\item Program Committee, Concur 2005.
\item Program Committee, Interational Conference on Supercomputing 2005.
 \item Program Committee, Workshop on Productivity in High Performance computing, 2005.
\item Editorial Advisor, Journal for Logic Programming till 2004.
 \item Program Committee, COLOPS 2004.
 \item Program Committee, Computational Methods in Systems Biology, 2004.
 \item Chair, Program Committee, ASIAN 2003. Proceedings to be
   published by Springer Verlag.
 \item Session Organizer for ``Web services'', Second Workshop,
   Venice, Oct 2003. Invited by the Steering Committee for the CUE
   Initiative on The Scientific Foundation of Informatics as an
   Engineering Discipline.
 \item Co-organizer, Dagstuhl seminar on Concurrent Constraint
   Programming, 1997.
 \item Program Committee, ASIAN 1996.
 \item Program Committee, Algebraic and Logic Programming, 1996; 
 \item Program Committee, Principles of Programming Languages,  1996.
 \item Program Committee, Principles and Practice of Constraint
   Programming, 1994.
 \item Co-organizer and Co-Program Chair, Principles and Practice of
   Constraint Programming, 1993.
 \item Program Committee, Fifth Generation Computer Systems, 1992.
 \item Program Committee, Algebraic and Logic Programming,  1992; 
 \item Program Co-Chairman, International  Logic Programming Symposiumm, 1991;
 \item Program Committes, North American Logic Programming
   Conference, 1989, 1990;
 \item Program Committee, International Conference on Logic
   Programming, 1990,1991;
\end{itemize}


\subsection*{Professional Societies}


ACM (SIGPLAN). 

Past member of Association for Logic Programming, IEEE, American
Association for Artificial Intelligence, European Association for
Theoretical Computer Science, American Anthropological Association,
American Association for the Advancement of Science, SIAM.

\end{document}
\textit{ Last revised Sun Dec 15 06:40:38 2002}
\textit{ Last revised Sun May 04 14:11:53 2003}
\textit{ Last revised Sat May 31 22:40:19 2003}
\textit{ Last revised Tue Jul 26 08:43:03 2005}
\textit{ Last revised Thu May 08 07:19:00 2008}
\textit{ Last revised Fri May 01 08:09:55 2009} 
\textit{ Last Revised Mon Jan 20 07:08:37 EST 2014}
\textit{ Last Revised Fri Apr 15 06:37:19 EDT 2016}


