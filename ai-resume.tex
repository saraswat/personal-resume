\documentclass{article}
\def\mytitle{Vijay Saraswat Resume}
\def\mytitle{Vijay Saraswat Resume}
%\usepackage{html,htmllist}
\usepackage{epsfig}
%begin{latexonly} 
%\usepackage{url}
\usepackage{fancyheadings}
\setlength{\topmargin}{-0.46in}
\setlength{\oddsidemargin}{-0.0in}
\setlength{\evensidemargin}{-0.0in}
\setlength{\headheight}{.53in}
\setlength{\headsep}{.4in}
\setlength{\footskip}{.4in}
\setlength{\textheight}{8.30in}
\setlength{\textwidth}{6.30in}
\setcounter{tocdepth}{5}                                                        

\pagestyle{fancy}
\usepackage{lastpage209}
\headrulewidth 0.0in
\footrulewidth 0.0in
\rhead{}
\lhead{}
\rfoot{Page \thepage\ of \pageref{LastPage} Saraswat Resume}
\bigskip
\usepackage{ssiarticle}
%\end{latexonly}

\begin{document}
%\begin{htmlonly}
%<p align=right><a href="resume.pdf">printable pdf version</a></p>
%\end{htmlonly}


\section*{Natural language understanding}

\begin{itemize}
\item Introduction. (Mary Dalrymple, Fernando Pereira, John Lamping, Vijay
Saraswat) In \textit{Semantics And Syntax in Lexical Functional
Grammar: The Resource Logic Approach} edited by Mary Dalrymple. The
MIT Press, 1999. 

\item LFG qua Concurrent Constraint Programming. (Vijay Saraswat) In
    \textit{ Semantics And Syntax in Lexical Functional Grammar: The
    Resource Logic Approach} edited by Mary Dalrymple. The MIT Press, 1999.

\item Relating Resource-based semantics to categorial semantics. (Mary
    Dalrymple, Vineet Gupta, John Lamping, and Vijay Saraswat) In
   \textit{ Semantics And Syntax in Lexical Functional Grammar: The Resource
    Logic Approach} edited by Mary Dalrymple. The MIT Press, 1999.

\item A Deductive Account of Quantification in LFG.  (Mary
    Dalrymple, John Lamping, Fernando Pereira, and Vijay
    Saraswat) In \textit{ Quantifiers, Deduction, and Context}, ed.\
    Makoto Kanazawa, Christopher J.~Pi\~{n}\'{o}n, and Henriette
    de~Swart.  Stanford, California: Center for the Study of
    Language and Information.  1996. 

\item Andrew Kehler, Mary Dalrymple, John Lamping, and Vijay Saraswat
    The Semantics of Resource Sharing in Lexical-Functional Grammar.  
    Proceedings of the 1995 Meeting of the
    European Chapter of the Association for Computational
    Linguistics, Dublin, Ireland. March 1995.

\item Mary Dalrymple, John Lamping, Fernando Pereira, and Vijay Saraswat
    Intensional Verbs Without Type-Raising or Lexical Ambiguity.
    In Logic, Language and Computation, volume 1, ed. Jerry
    Seligman and Dag Westerstaahl. Stanford, California:
    Center for the Study of Language and Information.  1996.
    Also in Proceedings of the Conference on
    Information-Oriented Approaches to Logic, Language and
    Computation/Fourth Conference on Situation Theory and its 
    Applications, Saint Mary's College of California, Moraga,
    California. June 1994.

\item Dalrymple, Mary, John Lamping, and Vijay Saraswat. 
    LFG semantics via constraints. In Proceedings of the Sixth
    Meeting of the European ACL, University of Utrecht, April.
    European Chapter of the Association for Computational
    Linguistics, 1993.
    
\item Dalrymple, Mary, Angie Hinrichs, John Lamping, and Vijay Saraswat.
    The resource logic of complex predicate interpretation.
    In Proceedings of the 1993 Republic of China Computational
    Linguistics Conference (ROCLING), Hsitou National Park,
    Taiwan, September. Computational Linguistics Society of R.O.C.

\end{itemize}

\section*{Constraint solving}
\begin{itemize}
\item Danny Munera, Daniel Diaz, Salvador Abreu, Francesca Rossi, Vijay
Saraswat, Philippe Codognet,  "Solving Hard Stable Matching Problems
via Local Search and Cooperative Parallelization", AAAI 2015.

\item Parallel Combinatorial Optimization with Decision Diagrams
David Bergman, Andre Cire, Ashish Sabharwal, Horst Samulowitz, Vijay
Saraswat, Willem-Jan van Hoeve (2014)  CPAIOR-2014. 11th International
Conference on Integration of AI and OR Techniques in Constraint
Programming, Cork, Ireland, May 2014.  

\item SatX10: A Scalable Plug\&Play Parallel SAT Framework 
Bard Bloom, David Grove, Benjamin Herta, Vijay Saraswat, Ashish
Sabharwal, and Horst Samulowitz (2012).  Published at Proceedings of
the 15th International Conference on Theory and Applications of
Satisfiability Testing (SAT 2012)   
\end{itemize}

\section*{Scientific and Engineering reasoning and problem-solving}

\begin{itemize}
\item 
\item Olivier Raiman, Johan de Kleer, Vijay Saraswat "Critical Reasoning",
IJCAI 1993 

\item Olivier Raiman, Johan de Kleer, Vijay Saraswat, Mark Shirley
"Characterizing Non-intermittent faults", AAAI 1991

\item Saraswat, V.A. and de Kleer, J. and Raimon, O.
  ``Contributions to the theory of diagnosis'', International
  Workshop on Principles of Diagnosis, Stanford University, July
  1990.
\end{itemize}

\end{document}
