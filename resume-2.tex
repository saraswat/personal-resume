\documentclass{article}
\def\mytitle{Vijay Saraswat Resume}
%\usepackage{html,htmllist}
\usepackage{epsfig}
%begin{latexonly}
%\usepackage{url}
\usepackage{fancyheadings}
\setlength{\topmargin}{-0.46in}
\setlength{\oddsidemargin}{-0.0in}
\setlength{\evensidemargin}{-0.0in}
\setlength{\headheight}{.53in}
\setlength{\headsep}{.4in}
\setlength{\footskip}{.4in}
\setlength{\textheight}{8.30in}
\setlength{\textwidth}{6.30in}
\setcounter{tocdepth}{5}                                                        

\pagestyle{fancy}
\usepackage{lastpage209}
\headrulewidth 0.0in
\footrulewidth 0.0in
\rhead{}
\lhead{}
\rfoot{Page \thepage\ of \pageref{LastPage} Saraswat Resume}
\bigskip
\usepackage{ssiarticle}
%\end{latexonly}

\begin{document}
%\begin{htmlonly}
%<p align=right><a href="resume.pdf">printable pdf version</a></p>
%\end{htmlonly}

\begin{center}
Vijay Anand Saraswat\\
Mountain Lakes, NJ\\
\quad\\
{\bf Web:} www.saraswat.org\\
{\bf Email:} vj@saraswat.org\\
{\bf IM:} (AOL, Yahoo, AT\&T) vjSaraswat\\
{\bf IM:} (MSN) vj@saraswat.org\\
\end{center}

\section*{Education}
\begin{itemize}
\item PhD, Computer Science, Carnegie Mellon University, January 1989;
   Title: Concurrent Constraint Programming;
   Advisor: Dana S. Scott.

\item M.S., Computer Science, Carnegie Mellon University, 1985.

\item B.Tech. Electrical Engineering, Indian Institute of Technology,
   Kanpur, India, 1982.
\end{itemize}

\section*{Employment and other work}

\begin{itemize}

%%\item January 2003 -- onwards: {\em Academic Visitor}, IBM Research,
%% Yorktown Heights. (Concurrent with Penn State position.)

\item August 2002 -- onwards: {\em Professor of Computer Science and
Engineering}, Penn State University

 Area: Programming, Languages and Systems.

\item June 2001 -- August 2002: {\em Vice President, Engineering}, Kirusa.

  Responsible for recruiting the engineering team, developing product
  plans, architecting the product, and delivering the product.
  Product to be trialed with carrier customers in Europe in June
  2002. Work involves development and implementation of XML-based
  Multimodal markup language, for delivery over 2.5G (GPRS)
  networks. Sophisticated integration work, involving WML, VoiceXML,
  HTML, SALT, XML technologies, delivered using a J2EE infrastructure,
  and native PocketPC clients that process speech. Doing significant
  background work on understanding distributed speech recognition, and
  efficient delivery of (possibly processed) speech over GPRS.

  Also responsible for the company's Applications group, Product
  Delivery group, and the company's IT infrastructure. 

  Since August 2002, continuing as a Consultant to Kirusa.

\item December 2000 - June 2001: {\em Chief Technology Officer}, Vayusphere.

  Responsible for recruiting engg team in San Diego and Mountain View
  (Dev Directors, Architects, Engineers; QA Staff).

  Responsible for defining, architecting and shipping first set of
  products (Monsoon Real-time Messaging Platform), setting the
  technical direction for the company, evaluating vendor technology,
  and partnering with business development.

  Responsible for the ``vision sell'' which brought in the company's
  first major customer.

  Shipped product end of May 2001.
  
\item July 2000 - December 2000:  {\em Director, Development}, Vayusphere.

  Employee \#6 at Vayusphere. Designed, directed and architected the
  implementation of Vayusphere Instant Messaging. (See ``Systems
  work'' section below for description of work.)

  Designed and implemented agent architecture used to establish
  client-side extensibility (e.g. integration with Outlook/Exchange)
  and buddy services. The success of this work caused the Board to
  re-center the company around (Wireless) Instant Messaging.
  
  Named on the Times Digital ``Movers and Shakers for 2001'' list (Nov
  2000) for Internet Messaging.
  
\item November 99 - July 2000: 
  {\em Technology Leader, Instant Messaging and Presence}, AT\&T Shannon
  Labs.

  Secured commitment from AT\&T Wireless to trial AT\&T Instant
  Messaging for their Pocket Net service. 

  Grew the Labs Instant Messaging team. At its peak, the work
  involved over 40 people, with about 15 managers.

  Continued strategic work for AT\&T leadership on Instant
  Messaging. 

  Continued as Co-chair of IETF IMPP WG. Led industry discussions,
  culminating in the formation of the IMUnified group including MSN,
  Yahoo, AT\&T, and other players. Oversaw initial development of the
  IMUnified protocol.

\item November 98 - November 99: 
  {\em District Manager, Network Community Platform Group}, AT\&T
  Shannon Labs

  Led the design, architecture and development of AT\&T Instant
  Messaging, based on Matrix. Formed the team from scratch. (See
  description of System work below.)

  Managed the growth of the team from a small research-based group to
  a production group incorporating standard AT\&T practices in
  requirements development, system testing, MR management and
  operations management.

  Leveraged an offshore implementation team in India to manage
  low-cost, round-the-clock development efforts.

  Provided strategic input on Instant Messaging to members of the
  AT\&T Operations Group (President of AT\&T Labs, President of
  AT\&T Corporate Strategy, and Chief Counsel for AT\&T). Led
  discussions with industry on interoperable Instant Messaging.

  Obtained \$10m for Instant Messaging and Presence work for 2000
  from the AT\&T Operations Group (chaired by Mike Armstrong), as
  part of the IP Platform work.
  
  Obtained funding from AT\&T Consumer Services for a Trial of AT\&T
  Instant Messaging.

  Co-Chaired the IETF Working Group on Instant Messaging and Presence
  Protocols (IMPP). All the major industry players -- Microsoft,
  Yahoo, AOL -- have agreed to support the work of this industry
  group.
   
  Industry spokesperson for national media (WSJ, USA Today, NY Times,
  Washington Post, SJ Mercury News ...) on Instant Messaging.

\item September 96 - November 98: 
  {\em Principal Member, Technical Staff}, AT\&T Shannon Labs
 
  Developed and implemented the underlying ideas for Matrix, an
  architecture for extensible network spaces based on Java. 

  Java evangelist within AT\&T. 

  Identified a major security bug in the design for Java class-loaders
  in JDK 1.1, and worked with Java designers at JavaSoft (Gilad
  Bracha, Sheng Liang) to fix the bug. See their paper in OOPSLA 99.

  Ran Meadows, a school-centered network community (base on MOO
  technology) involving multiple school districts in New Jersey
  and abroad.

\item November 87 --- September 96: 
  {\em Member, Research Staff}, Xerox Palo Alto Research Center.

  Developed the notion of concurrent constraint programming
  languages.
  
  Initiated and led the thrusts at PARC in model-based
  computing, hybrid computing and network communities.

  (See description of research work below.)

\item April 86 --- November 87: {\em Artificial Intelligence
  Scientist}, Advanced Product Developments Group, Carnegie Group Inc.

  Helped initiate the ``Carnegie Inference Language'' project,
  to develop the next generation of expert-system development
  tools.  Co-led language design, which was based on ideas from
  production systems, concurrent logic programming and
  object-oriented programming.

\item January 85 --- March 86: {\em Consultant}, Carnegie Group Inc, Pittsburgh.
 Designed and implemented CRL-Prolog.
\end{itemize}

\section*{Academic awards and Industry recognition}

\begin{description}

\item{2002:} Invited Expert of the W3C Working Group on Multi-Modal
Interaction.

\item{2001:} Co-Chair of the WG on Presence and Instant Messaging
(PRIM), IETF.

\item{1998-2000:} Co-Chair of the WG on Instant Messaging and
 Presence Protocols (IMPP), IETF.

\item{1994:} Excellence in Research Award, Xerox PARC.   

\item{1992:} Excellence in Support of Research Award, Xerox PARC.   

\item{1989:} ACM Doctoral Dissertation Award for the best Computer Science
   PhD Thesis in 1989. Thesis published by MIT Press.

\item{1982:} Ratan Swaroop Gold Medal for all-round excellence,
      Indian Institute of Technology, Kanpur. 

\item{1982:} Scholarship from the Inlaks Foundation for graduate studies
      at the Programming Research Group at Oxford; Fellowship
      offers from Univ. of  Minnesota; scholarship offer from 
      Carnegie-Mellon, Wisconsin-Madison, Brown, etc.  in
      Computer Science.  

\item{1982:} Second prize in All-India Student Paper Contest of the
      Computer Society of India for  Design and implementation of the
      C-code abstract machine for CCN-PASCAL. 

\item{1980:} Merit Scholarship for being the top-ranked student in the
      ``core'' (first 2.5/5) years  at IIT. 

\item{1977:}  Award for ranking in the first 35 (out of over 50,000) in 
      the nation in the All-India Higher Secondary (school-leaving)
      examination. Awarded a National Merit Scholarship by the
      Government of India. 

\item{1977:} Ranked 6th (out of $> 60,000$) in the Joint Entrance Exam
      for all the IITs.  

\item{1977:} Awarded one of five scholarships from India for
      a Baccalaureate at the United World College, Singapore.  
\end{description}

\section*{Research Grants}
\begin{itemize}
\item   Co-Principal Investigator (with Danny Bobrow), ``Testing of hybrid
   and reactive systems'', 1 yr, \$150K NASA, 1994-1995. (Xerox PARC)

\item   Principal Investigator, ``Timed Concurrent Constraint
   Programming'', 2 yrs, \$120K, ONR, 1994--1996.  (Xerox PARC)

\item   Co-Principal Investigator (with Danny Bobrow), ``Articulate Spaces:
   Model-based authentic environments for Collaborative Learning'', 2
   yrs, \$2m ARPA, Department of Defense, 1995-1997.

\item   Co-Principal Investigator (with Danny Bobrow, Billie Hughes, Jim
   Walters), ``Collaborative learning spaces'', NSF, 1 year, 1996-97.
\end{itemize}

In addition, in 1991 Seif Haridi (Director, Swedish Institute of
Computer Science, Stockholm) and I conceived of the {\sc ACCLAIM} project
(Advanced Concurrent Constraint Languages --- Applications,
Implementation and Methodology). The project was funded for several
million ECUs by ESPRIT, the European Community Research Funding
Agency, and involved the Max-Planck Institut and DEC Paris Research
Lab, INRIA, DFKI, SICS, RISC Linz, Universidad Politecnica de Madrid,
Universita di Pisa, Marseille Luminy and Katholieke Universiteit
Leuven. The list of deliverables from the project at
\begin{verbatim}
  http://www.sics.se/ps/acclaim/deliverables/perpartner.html
\end{verbatim}
enumerates approximately a hundred papers on concurrent constraint
programming.

\section*{Instruction}
\begin{itemize}
  \item CS 428 Programming Languages, Penn State, Fall 2002.
  \item CS 598f ``Concurrent Constraint Programming as a Foundation
  for Model-based Programming'', Graduate course, Penn State, Spring
  2003.
\end{itemize}
\section*{Theses}
\subsection*{Thesis committee member}
\begin{itemize}
\item  Francesca Rossi (PhD 1993, University of Pisa) (Advisor: Ugo Montanari)
\item  Eric Torng (PhD 1994, Stanford University) (Advisor: Rajeev Motwani)
\item  Paul Ruet (PhD 1996, University of Paris) (Advisor: Francois Fages)
\end{itemize}

\subsection*{Thesis adviser}
\begin{itemize}

\item  Clifford Tse, ``Linear Janus'' (Master's thesis 1992, MIT Laboratory for
  Computer Science) MIT-PARC VI-A student

\end{itemize}

\section*{Books}
\begin{itemize}
\item   Vijay Saraswat ``Concurrent Constraint Programming'' MIT
  Press Logic Programming and Doctoral Dissertation Award Series,
  1991.

\item   Proceedings of the 1991 International Symposium on Logic
  Programming, ed.{} Vijay Saraswat and Kazunori Ueda, MIT Press,
  1991.

\item  Constraint Programming: The Newport Papers, ed.{} Vijay
  Saraswat and Pascal van Hentenryck, MIT Press, 1995.
\end{itemize}
\section*{Edited journal issues}
\begin{itemize}
\item  Vijay Saraswat, Pascal van Hentenryck Special issues of
  CONSTRAINTS on Strategic Directions in Constraint Programming,
  February 1997.

\item  Pascal van Hentenryck, Vijay Saraswat, ed. ``Constraint
  Programming'', in Special Issue of Computing Surveys on
  Strategic Directions in Computer Science, February 1997.
\end{itemize}

\section*{Systems work}
\begin{description}

\item[1986]{\bf CRL-Prolog, CRL-OPS}

While a scientist at Carnegie Group Inc., designed and implemented
  CRL-Prolog --- a fast Prolog implementation that compiles into
  Common Lisp and is closely integrated with the frame-based language
  CRL.  Supervised the implementation of CRL-Ops.  CRL-Prolog
  and CRL-Ops were shipped as part of the product {\em Knowledge
  Craft}.

\item[1992]{\bf QD-Janus}

Collaborated with Saumya Debray on the QD Janus implementation. See
  ``S.K.~Debray, QD-Janus: A sequential implementation of Janus in
  Prolog, Software Practice and Experience, Volume 23, Number 12,
  December 1993, pp. 1337-1360.''

\item[1992-1994]{\bf Model-based Computing.} 

While at Xerox PARC, initiated and led a project on constraint-based
  machine control.  Implemented the first version of a
  constraint-based real-time scheduler in C++, and supervised a small
  team of engineers at Xerox Engineering in Rochester (New York).
  Code was shipped in the embedded controller for Xerox' digital
  mid-volume reprographics engines.  This work led to several US
  Patents, and an award for Excellence in Research from Xerox PARC.

\item[1995]{\bf Pueblo.}

Founded the Pueblo online community (pueblo.xerox.com 7777), using the
 MOO system, with Jim Walters and Billie Hughes of Phoenix College,
 January 1995. As of May 1996, the community had over 1000 characters,
 including over 300 students from Phoenix and New York, and has over
 20,000 objects.

 Responsible for running the MOO, and creating many world objects.

\item[1997]{\bf Meadows.}

 Founded the Meadows online community for parents, school-children and
 teachers across multiple school districts. Responsible for running
 the MOO, and extending it with various objects.

 Taught classes in elementary and middle school (2d grade to 6th
 grade) on Meadows.

\item[1997-1998]   {\bf AT\&T Matrix.}

  Built the first version of the Matrix system in
  Java 1.1.  AT\&T Matrix was a persistent network community server that
  could be extended by programming world objects in Java.  A Matrix
  server is accessed through a Matrix client (developed in Java using
  Swing).  Servers and clients communicate through a completely
  self-contained {\em session-oriented Remote Method Invocation} (SMI)
  system designed and implemented after analyzing the very poor design
  of the initial Java RMI system.  SMI uses Java's native object
  serialization and class loaders.  World objects provided their own
  graphical user interface, displayed in clients.

  Supervised development of a Java Swing-based client, and the
  development of several world objects (games, bulletin boards, chat
  rooms).  Core team grew to approx. half a dozen people, before
  project was converted into an Instant Messaging project (see below).

  The system was demonstrated to several people in the Labs, was in
  use by the development team, and served as the basis for an internal
  Instant Messaging trial for several hundred people in AT\&T Labs in
  summer 1999.

  The success of this work led to the invitation to form the Network
  Communities Platform Group (NCPG) in AT\&T Labs in November 1998.

\item[1999-2000] {\bf AT\&T Instant Intercom.} 

  Founded, staffed and directed the development team that delivered
  AT\&T's instant messaging system into trial with AT\&T Wireless in
  March 2000. I was the main designer and architect, and chief
  server-side implementer. After I left AT\&T, the system was deployed
  by AT\&T Worldnet. In production use today, it has been downloaded
  by over 120,000 users, and supports approx 20,000 simultaneous
  users. It has been designed to scale to a million registered and
  100,000 simultaneous users.

  The system features a modular architecture with several different
  kinds of servers (IM servers, Presence servers, Chat room
  servers). I designed the binary extensible protocol (documented in
  an IETF Draft) used by servers to communicate with each other and
  with clients. Supervised the design of a C++ I/O module to
  circumvent scalability limitations of Java's synchronous I/O
  architecture.

  Supervised development of a native Windows client in Delphi, a Java
  applet client, a black-phone voice client (using PML, a precursor to
  VoiceXML), and of an HDML client (in collaboration with
  AT\&T Wireless).

\item[2000-2001] {\bf Vayusphere Monsoon Real-time Messaging System}.

Director of the group, designer and chief server-side implementer.
  Monsoon features a scalable, open, extensible architecture for
  Instant Messaging, with two-way interop with email. All server-side
  code was in Java 2; the system also used an Oracle Database, and
  made extensive use of open source components (Apache, Tomcat
  (servlet runner), James (mail engine)). Designed a simple wire
  protocol which is used for all server-to-server and client-to-server
  communication.

  Supervised development of native clients in C and C++ for the RIM
  Mobitex devices, wireless Palm OS devices, and wireless BREW
  devices.  Supervised development in Delphi of a full-featured
  Windows desktop client. Supervised development of HDML/WAP clients
  for phones.

  Assembled the engineering and QA team from scratch, and had the
  product ready for beta in under three months.  Development and QA
  teams were scattered over Mountain Lakes (NJ), Mountain View (Ca),
  San Diego (Ca), Oakland (Ca) and Indiana, and at peak numbered
  approx 20 people.

  Product released to first customer (major national ISP) in nine
  months.


 \item[2001-2002] {\bf Kirusa Multimodal Platform.}

 As VP Engineering, leading the team that is developing and delivering
 KMMP.  Chief architect for the product, which supports the development
 of multimodal applications (those which simultaneously use voice and
 visual interfaces) for wireless devices. The product is primarily
 based on Java proxy/servlet technology and is designed for 2.5G
 networks (e.g.{} GPRS), and processes the XML-based multimodal markup
 language (M3L), which integrates the WAP Markup Language (WML) and the
 voice markup language (VoiceXML).

 Chief designer of M3L, and author of the language spec.

 \item[2002-onwards] {\bf The Java concurrent constraint programming
system} Implementing the {\sf jcc} system under the Lesser GPL
licence on SourceForge. The system implements the (default) (timed)
concurrent constraint programming framework in Java, for embedded
and discrete- and continuous-time computing. The implementation is
targeted for model-based programming applications in the NASA domain
(Mars rovers), and in the systems biology domain.

\end{description}

\section*{United States patents}

Several patent applications have been filed by AT\&T based on my work
in Instant Messaging.

\begin{itemize}
\item  Patent pending on ``Wireless call waiting''.
\item  Patent pending on ``Object-enabled Instant Messaging''.
\end{itemize}

Patents resulting from model-based scheduling work at Xerox PARC:
\begin{itemize}
\item   US05831853, 11/98, Automatic construction of digital
  controllers/device drivers for electro-mechanical systems using
  component models

\item   US05701557 12/97 Machine graphs and capabilities to represent
  document output terminals composed of arbitrary configurations

\item  US05696893 12/97 System for generically describing and
  scheduling operations of modular printing machine

\item  US05631740 5/97 Transducers with constraints model for print
  scheduling

\item  US05617214 4/97 Commitment groups to generalize the scheduling
  of interdependent document output terminal capabilities

\item  US05504568 4/96 Print sequence scheduling system for duplex
  printing apparatus
\end{itemize}


\section*{Publications}
\subsection*{Multi-modal Systems}
\begin{itemize}
\item Stephane Maes and Vijay Saraswat ``Multi-Modal Requirements'', 
W3C Note, January 2003.
\end{itemize}				   

\subsection*{Network communities}
\begin{itemize}
\item Vijay Saraswat and Fernando Pereira ``Interaction media: Some
  thoughts on models for cyberspace'', Proceedings of the Virtual
  Worlds in Simulation Conference, San Francisco, January 1999.

\item  Vijay Saraswat ``Design requirements for network spaces'',
  Proceedings of the Virtual Worlds in Simulation Conference, San
  Francisco, January 98.

\item Jay Carlson, Roger Crew, Ken Fox, Richard Goddard, Dave Kormann,
  Erik Ostrom, John Ramsdell, Vijay Saraswat, Andrew Wilson ``The MUD
  Client Protocol, Version 2.1'', http://www.moo.mud.org/mcp2.

\item  Vijay Saraswat ``The dog, the catcher, the fish and the
  frying pan: Melding work, play and theater in network
  community'', Virtual Communities 97, February 1997, Sydney,
  Australia. 
  
\item  Vicki O'Day, Daniel Bobrow, Billie Hughes, Kimberly Bobrow,
  Vijay Saraswat, JoAnne Talazus, Jim Walters, Cynde Welbes
  ``Community Designers'', Participatory Design Conference, 1996.

\item  Daniel Bobrow, Vicki O'Day, Vijay Saraswat, Billie Hughes and
  Jim Walters ``Learning through computationally-mediated
  conversations: Interaction and Construction in virtual
  spaces'', Presented at the Annual Meeting of the American
  Anthropological Association, Washington D.C., November 1995.
\end{itemize}
\subsection*{Foundations for Capability Programming}

\begin{itemize} 
\item Vijay Saraswat and Radha Jagadeesan ``Static support for
  capabilities in Java'', Second APPSEM workshop, U of Nottingham,
  April 2003.
\end{itemize}

\subsection*{Semantic foundations for concurrent programming}
\begin{itemize} 

\item  Vineet Gupta, Radha Jagadeesan and Vijay Saraswat ``Probabilistic
  Concurrent Constraint Programming'', Proceedings of the
  International Conference on Concurrency Theory, LNCS, CONCUR '97,
  243-257.

\item  Vineet Gupta, Radha Jagadeesan and Vijay Saraswat ``Models
  of concurrent constraint programming'', Proceedings of the
  International Conference on Concurrency Theory, LNCS 1119 August
  1996.

\item Vineet Gupta, Radha Jagadeesan and Vijay Saraswat ``Truly
  Concurrent Constraint Programming'', Theoretical Computer Science,
  Volume 278, pp 223-255, 2002. Conference version in Proceedings of
  the International Conference on Concurrency Theory, LNCS 1119 August
  1996.

\item  Ugo Montanari, Francesca Rossi, and Vijay Saraswat ``Event
  structure semantics for concurrent constraint programming'',
  1994.  
  
\item  Ugo Montanari, Francesca Rossi, and Vijay Saraswat ``CC
  programs with both in- and non-determinism'', 1994.

\item  Prakash Panangaden, Vijay Saraswat, Phillip J. Scott and
  Robert Seely, ``A Hyperdoctrinal view of concurrent constraint
  programming'', Proceedings of the REX Workshop on Semantics:
  Foundations and Applications, eds. J. W. deBakker, W.-P. de Rover
  and G. Rozenberg, LNCS 666, pp 457-476, 1993.
  
\item  Vijay Saraswat ``The category of constraint systems is
  Cartesian-closed'', Proceedings of the Symposium on Logic in
  Computer Science, Santa Cruz, June 1992.

\item Radha Jagadeesan, Vijay Saraswat and Vasant Shanbhogue ``Angelic
  non-determinism in concurrent constraint programming'', Technical
  Report, Xerox PARC, 1992.
  
\item  Vijay Saraswat and Rinard M. and Panagaden, P. ``Semantic
  foundations for concurrent constraint programming'', {\em
  Proceedings of the ACM Symposium on Principles of Programming
  Languages}, Orlando, January 1991.
  
\item Patrick Lincoln and Vijay Saraswat ``Proofs as concurrent
  processes: A logical interpretation for concurrent constraint
  programming'', Technical Report, Systems Sciences Laboratory, Xerox
  PARC, November 1991. (Revised report, Penn State University, April
  2003.)
  
\item  Vijay Saraswat and Rinard, M. ``Concurrent constraint
  programming'', Proceedings of the ACM Symposium on
  Principles of Programming Languages 1990, San Fransisco, January
  1990.
  
\item  Vijay Saraswat ``A somewhat logical formulation of CLP
  synchronization primitives'', Proceedings of Logic Programming,
  August 1988, MIT Press.
  
\item  Vijay Saraswat ``The language {\bf CP}: Definition and
  Operational semantics'', in Proceedings of the ACM
  SIGACT-SIGPLAN Conference on Principles of Programming
  Languages, Munich, January 1987.
  
\item  Vijay Saraswat ``{\sf CP} as a general-purpose
  constraint-language'', Proceedings of the National
  Conference on Artificial Intelligence, (AAAI), Seattle, July
  1987.
  
\item  Vijay Saraswat ``The language GHC: operational semantics
  and comparison with {\bf CP({\tt !},{\tt |})}'', Proceedings of the
  Fourth IEEE Symposium on Logic Programming, San Fransisco,
  September 1987.
  
\item  Vijay Saraswat ``Partial Correctness semantics for {\bf
  CP({\tt !},{\tt |}, {\tt \&})}'', Proceedings of the Conference on
  Foundations of Software Technology and Theoretical Computer
  Science, Springer Verlag LNCS 206, pp. 347-368, December 1985.
\end{itemize}
\subsection*{Real-time and hybrid systems}
\begin{itemize} 
\item  Vineet Gupta, Radha Jagadeesan and Vijay Saraswat
  ``Computing with Continuous Change'',  Science of
  Computer Programming, 30 (1:2) 3-49, 1998.

\item  Vijay Saraswat, Radha Jagadeesan and Vineet Gupta
  ``Timed Default Concurrent Constraint Programming'', In Journal of
  Symbolic Computation 22 (5,6) 475--520, 1996. Extended abstract
  published in the Proceedings of the ACM Symposium on Principles of
  Programming Languages, San Fransisco, 1995.
  
\item  Vineet Gupta, Radha Jagadeesan and Vijay Saraswat
  ``Hybrid CC, Hybrid Automata, and Prgram Verification'', Hybrid
  Systems Workshop, DIMACS, Rutgers, October 1995. Appeared in Hybrid
  Systems III Verification and Control ed.{} R. Alur, T.A.{}
  Henzinger, E.D.{} Sontag (LNCS 1066), Springer-Verlag, Berlin, 1996.

\item  Vineet Gupta, Radha Jagadeesan, Vijay Saraswat, and Daniel
  Bobrow ``Programming in Hybrid Constraint Languages'', Hybrid
  Systems Workhsop, Cornell, October 1994. Hybrid Systems II, ed.{}
  P.~Antsaklis, W.~Kohn, A.~Nerode, S.~Sastry (LNCS 999),
  Springer-Verlag, Berling, 1995.

\item  Vijay Saraswat, Radha Jagadeesan and Vineet Gupta
  ``Foundations of Timed Concurrent Constraint Programming'',
  Proceedings of the Symposium on Logic in Computer Science, Paris,
  July 1994.
  
\item  Vijay Saraswat, Radha Jagadeesan and Vineet Gupta
  ``Programming in Timed Concurrent Constraint Programming'',
  Chapter in Constraint Programming, ed.{} B. Mayoh and E. Tyugu,
  NATO ASI Workshop, April 1994.
\end{itemize}

\subsection*{Modeling and diagnosis}
\begin{itemize} 
\item  Markus Fromherz, Vijay Saraswat and Daniel G. Bobrow
  ``Model-based computing: Developing flexible machine control
  software'', Artificial Intelligence, 114(1-2): 157-202 (1999)

\item  Yumi Iwasaki, Adam Farquhar, Vijay Saraswat, Daniel Bobrow
  and Vineet Gupta ``Modeling time in hybrid systems: How fast is
  ``instantaneous''?'', Proceedings of the International Joint
  Conference on Artificial Intelligence, Montreal, August, 1995.
  
\item  Vineet Gupta and Vijay Saraswat and Peter Struss, ``A model
  of a photocopier paper path'', Proceedings of the 2nd IJCAI
  Workshop on Engineering Problems for Qualitative Reasoning,
  August 1995.

\item Hao-Chi Wong and Markus P.J. Fromherz and Vineet Gupta and Vijay
  A. Saraswat. ``Control-based programming of electro-mechanical
  controllers.'' Proceedings of the IJCAI Workshop on Executable
  Temporal Logics, Montreal, August 1995.

\item  Markus Fromherz and Vijay Saraswat   ``Model-based
  computing: using concurrent constraint programming for modeling
  and model compilation'', U. Montanari  and F. Rossi (ed.)
  Principles and Practices of Constraint Programming, CP'95,
  Springer-Verlag, LNCS 976, Sep 1995, pp. 629--635. 
  
\item  Markus Fromherz and Vijay Saraswat ``Model-based
  computing: constructing constraint-based software for
  electro-mechanical systems'' Practical Applications of
  Constraint Technology, Paris, France, April 1995, pp. 63-66.

\item  Markus Fromherz, David Bell, Daniel Bobrow, Brian
  Falkenhainer, Vijay Saraswat and Mark Shirley ``{\sc Rapper}:
  The Copier Modeling Project'', Working Papers of the Eight
  International Workshop on Qualitative Reasoning about physical
  systems'', pages 1-12, June 1994. 

\item Olivier Raiman and Johan de Kleer and Vijay Saraswat ``Critical
  reasoning'', Proceedings of the International Joint Conference on
  Artificial Intelligence, 1993.

\item P. Codognet and V. Saraswat, ``Abduction in Concurrent
  Constraint Languages'', Proceedings of the First Compulog Network
  meeting on Logic Programming and Artificial Intelligence, London,
  U.K., 1992.

\item  Olivier Raiman and Johan de Kleer and Vijay Saraswat and
  Mark Shirley ``Characterizing non-intermittent faults'',
  Proceedings of the National Conference on Artificial
  Intelligence, June 1991.
  
\item  Saraswat, V.A. and de Kleer, J. and Raimon, O.
  ``Contributions to the theory of diagnosis'', International
  Workshop on Principles of Diagnosis, Stanford University, July
  1990.
\end{itemize}
\subsection*{Natural language}
\begin{itemize}

\item Introduction. (Mary Dalrymple, Fernando Pereira, John Lamping,
   Vijay Saraswat) In {\em Semantics And Syntax in Lexical Functional
   Grammar: The Resource Logic Approach} edited by Mary Dalrymple. The
   MIT Press, 1999.

\item LFG qua Concurrent Constraint Programming. (Vijay Saraswat) In
   {\em Semantics And Syntax in Lexical Functional Grammar: The Resource
    Logic Approach} edited by Mary Dalrymple. The MIT Press, 1999.

\item LFG qua Concurrent Constraint Programming. (Vijay Saraswat) In
   {\em Semantics And Syntax in Lexical Functional Grammar: The Resource
    Logic Approach} edited by Mary Dalrymple. The MIT Press, 1999.

\item Relating Resource-based semantics to categorial semantics. (Mary
    Dalrymple, Vineet Gupta, John Lamping, and Vijay Saraswat) In
   {\em Semantics And Syntax in Lexical Functional Grammar: The Resource
    Logic Approach} edited by Mary Dalrymple. The MIT Press, 1999.

\item    A Deductive Account of Quantification in LFG.  (Mary
    Dalrymple, John Lamping, Fernando Pereira, and Vijay
    Saraswat) In {\em Quantifiers, Deduction, and Context}, ed.\
    Makoto Kanazawa, Christopher J.~Pi\~{n}\'{o}n, and Henriette
    de~Swart.  Stanford, California: Center for the Study of
    Language and Information.  1996. 

\item    Andrew Kehler, Mary Dalrymple, John Lamping, and Vijay Saraswat
    The Semantics of Resource Sharing in Lexical-Functional Grammar.  
    Proceedings of the 1995 Meeting of the
    European Chapter of the Association for Computational
    Linguistics, Dublin, Ireland. March 1995.

\item    Mary Dalrymple, John Lamping, Fernando Pereira, and Vijay Saraswat
    Intensional Verbs Without Type-Raising or Lexical Ambiguity.
    In Logic, Language and Computation, volume 1, ed. Jerry
    Seligman and Dag Westerstaahl. Stanford, California:
    Center for the Study of Language and Information.  1996.
    Also in Proceedings of the Conference on
    Information-Oriented Approaches to Logic, Language and
    Computation/Fourth Conference on Situation Theory and its 
    Applications, Saint Mary's College of California, Moraga,
    California. June 1994.

\item    Dalrymple, Mary, John Lamping, and Vijay Saraswat. 1993.
    LFG semantics via constraints. In Proceedings of the Sixth
    Meeting of the European ACL, University of Utrecht, April.
    European Chapter of the Association for Computational
    Linguistics. 
    
\item    Dalrymple, Mary, Angie Hinrichs, John Lamping, and Vijay Saraswat.
    The resource logic of complex predicate interpretation.
    In Proceedings of the 1993 Republic of China Computational
    Linguistics Conference (ROCLING), Hsitou National Park,
    Taiwan, September. Computational Linguistics Society of R.O.C.
\end{itemize}
\subsection*{Concurrent programming languages and paradigms}
\begin{itemize} 

\item Vijay Saraswat ``Java is not type-safe'', Web-note
  \verb+http://www.research.att.com/~vj/bug.html+. Described a major
  bug in the design of the class-loader mechanism. This was
  acknowledged and fixed by Sun in a major redesign. See Bracha and
  Liang's paper in OOPSLA 98.

\item Pascal van Hentenryck, Yves Deville, Vijay Saraswat ``Design,
    implementation and evaluation of the constraint language cc(FD)'',
    Journal Of Logic Programming 37(1-3):139-164 (1998). Conference
    paper in LNCS 910, pp 293-316 (1994).


\item    Vijay Saraswat and Patrick Lincoln ``Higher-order, linear
    concurrent constraint programming'', Xerox PARC Technical report,
    August 1992.

\item    Vijay Saraswat and Kenneth Kahn and Jacob Levy ``{\sf
    Janus}: A step towards distributed constraint programming'', {\em
    Proceedings of the North American Conference on Logic
    Programming}, Austin, Texas, October 1990.

\item    Kenneth Kahn and Vijay Saraswat ``Actors as a special
    case of concurrent constraint (logic) programming'', {\em
    Proceedings of the ECOOP/OOPSLA conference, 1990}.
  
\end{itemize}

The {\sf Janus} work has significantly influenced the development of
the (distributed, secure) language {\tt E}, currently being developed
by an open source group ({\tt www.erights.org}), and partly funded by
a DARPA grant to Mark S. Miller.

\subsection*{Concurrent programming: techniques, algorithms}
\begin{itemize} 
\item   Rajeev Motwani, Suresh Venkatsubramaniam, Rina Panigrahy, Vijay
   Saraswat ``On the decidability of accessibility problems'', ACM
   Symposium on the Theory of Computing, 2000.

\item   Eric Torng, Rajeev Motwani, and Vijay Saraswat ``Online
   scheduling with lookahead: Multipass assembly lines''
   INFORMS Journal on Computing, 1998.

\item   Saraswat, V.A. et al. ``Detecting stable properties
   of networks in concurrent logic programming languages'', in
   Proceedings of the ACM Conference on Principles of
   Distributed Computing, Toronto, August 1988.
  
\item   Vijay Saraswat ``Merging many streams efficiently: the
   importance of atomic commitment'', chapter in ``Concurrent
   Prolog: Collected Papers'', ed. E. Shapiro, MIT Press, December
   1987.
\end{itemize}

\subsection*{Visual programming}
\begin{itemize} 
\item    Kenneth M. Kahn and Vijay A. Saraswat ``Complete
    visualization of concurrent programs and their execution'',
    Proceedings of the IEEE Workshop on Visual Programming, October
    1990. 
\end{itemize} 

This work led to a rich body of work on Visual Programming, cf.{}
Pictorial Janus systems, and also to the company {\em Animated
Programs} founded by Ken Kahn. The company has introduced a
revolutionary product for school-children ``ToonTalk'', in the
tradition of Logo. See {\tt www.toontalk.com}.
    
\subsection*{Constraint programming in Software Engineering}
\begin{itemize} 
\item    Vijay Saraswat ``Compositional Computing'', CONSTRAINTS
    2(1):95-97 (1997)


\item Francesca Rossi and Vijay Saraswat ``Constraint Programming'',
in Encyclopedia of Computer Science and Technology (entry: Constraint
Programming), A. Kent and J.G.{} Williams eds, Marcel Drekker Inc,
1994.

\item    Markus Fromherz, Vineet Gupta and Vijay Saraswat, ``CC
     --- A  generic framework for domain specific languages'',
    Workshop  on Domain-oriented specification languages, POPL 97.
\end{itemize} 

\section*{Professional activities:}

\subsection*{Invited Panels:}
\begin{itemize}
\item  ``The Future of Constraint Programming'', CP 96, Boston, 1996.
\end{itemize}

\subsection*{Invited Presentations:}
\begin{itemize}
\item 
Penn State Logic Seminar, ``The logic of concurrent constraint
programming'', April 2003.

\item 
Invited participant, CUE Workshop on Component-based systems, Macau,
China SAR, October 2002.

\item IBM Research (T.J.~Watson Research Lab) ``Java as a metalinguistic
framework'', September 2002.
\item 
IBM Research (T.J.~Watson Research Lab) ``Java as a metalinguistic
framework'', September 2002.

\item 
IBM Research (T.J.~Watson Research Lab) ``The design of M''
October 2002.

\item
Chair, session on Multimodal Instant Messaging (Pulver Presence and
Instant Messaging Industry meeting), October 2001.

\item
Chair, session on Wireless Instant Messaging (Pulver Presence and
Instant Messaging Industry meeting), October 2000.

\item Invited Plenary Talk at ``Instant Messaging 2000'', Boston, May
    2000.

\item Tutorial on Concurrent Constraint Programming at CONCUR
    '96, Pisa Italy.

 \item Invited Talk at Workshop on Language, Logic and
    Information, CSLI, May 1996.

 \item Stanford University Theory Group, ``Hybrid Constraint
    Programming'', May 1996. 

 \item Distinguished Seminar in Programming Languages, U. of
    Chicago,  May 1996 ``Virtual World Programming Languages''. 

 \item Invited talk, First International Conference
    on Constraint Programming, Cassiss, France, September 1995. 

 \item Invited to attend Institute on Semantics of Programming
    Languages at Newton Institute, Cambridge, August 1995.  

 \item Keynote talk, First International Workshop on
    Concurrent Constraint Programming, Venice, May 1995. 

 \item Invited Talk on Timed Concurrent Constraint
    Programming, Royal Institute of Mathematics, Kyoto, August
    1994. 

 \item Workshop on Information Systems, Cosener House, Oxford,
    March 1992 (organized by Samson Abramsky and Bill Rounds).

 \item Sixth Italian Logic Programming Conference (GULP '91)
    on ``The Past as Prolog: Beyond Logic Programming''.

 \item Tutorial on ``The semantics of concurrent constraint
    programming languages'' at the North American Logic Programming
    Conference, Austin, Texas, October 1990.

 \item Tutorial on ``The paradigm of concurrent constraint
    programming'', at the Seventh  International Conference on Logic
    Programming, Jerusalem, June 1990. 

 \item ``Gigalips Workshop'', Swedish Institute of Computer
    Science, Stockholm, April 1989 (organized by Seif Haridi).

 \item Workshop on ``Languages and Constraints''  University of
    Rhode Island, April 1988 (organized by J. Cohen and J-.L.Lassez).

 \item AAAI-sponsored workshop on ``Concurrent Logic Programming
    and Open systems'', Xerox PARC, September 1987 (organized by Ken
    Kahn).

 \item Talks on various aspects of concurrent constraint programming
    at CMU, Cornell, MIT, Yale, UC Berkeley, Oxford University, U.~of
    Edinburgh, Imperial College, University of Texas at Austin, AT\&T
    Bell Labs, U.~of Pennsylvania, IBM T.J.~Watson Research Center,
    U.~of Utah, U.~of Arizona, U.~of Pisa, INRIA Versailles, Swedish
    Institute of Computer Science, Bristol Univ., Penn State
    University etc.
\end{itemize}

\subsection*{Refereeing responsibilities}
\begin{itemize}
\item  Journal of Logic Programming; 
\item  Artificial Intelligence; 
\item  Journal of the Association for Computing Machinery; 
\item  Theoretical Computer Science; 
\item  IEEE Computer;
\item  IEEE Software;
\item National Science Foundation (panelist Medium ITR 2003, reviewer)
\item IEEE Transactions on Computers;
\item  Various conferences including PLDI.
\end{itemize}

Additionally, papers reviewed for:
\begin{itemize}
\item ACM Programming Languages Design and Implementation Conference
(PLDI) 2003

\item ACM SIGPLAN Symposium on Languages, Compilers, and Tools for
Embedded Systems (LCTES) 2003
\end{itemize}

\subsection*{Editorial and Review Responsibilities: }
\begin{itemize}
  \item Editorial Advisor, Journal for Logic Programming. 
  \item Chair, Program Committee, ASIAN 2003. Proceedings to be
  published by Springer Verlag.
  \item Session Organizer for ``Web services'', Second Workshop,
  Venice, Oct 2003. Invited by the Steering Committee for the CUE
  Initiative on The Scientific Foundation of Informatics as an
  Engineering Discipline.
  \item Co-organizer, Dagstuhl seminar on Concurrent Constraint
    Programming, 1997.
  \item Program Committee, ASIAN 1996.
  \item Program Committee, Algebraic and Logic Programming, 1996; 
  \item Program Committee, Principles of Programming Languages,  1996.

  \item Program Committee, Principles and Practice of Constraint
  Programming, 1994.
  \item Co-organizer and Co-Program Chair, Principles and Practice of
  Constraint Programming, 1993.
  \item Program Committee, Fifth Generation Computer Systems, 1992.
  \item Program Committee, Algebraic and Logic Programming,  1992; 
  \item Program Co-Chairman, International  Logic Programming Symposiumm, 1991;
  \item Program Committes, North American Logic Programming
  Conference, 1989, 1990;
  \item Program Committee, International Conference on Logic
  Programming, 1990,1991;
\end{itemize}


\subsection*{Professional Societies:}

ACM (SIGPLAN, SIGACT, SIGART), Association for Logic Programming,
IEEE Computer Society, American Association for Artificial
Intelligence, European Association for Theoretical Computer Science, 
American Anthropological Association,
American Association for the Advancement of Science,
SIAM

\end{document}
{\em Last revised Sun Dec 15 06:40:38 2002}
{\em Last revised Sun May 04 14:11:53 2003}
{\em Last revised Sat May 31 22:40:19 2003}
